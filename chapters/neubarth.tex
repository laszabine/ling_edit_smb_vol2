%%%%%%%%%%%%%%%%%%%%%%%%%%%%%%%%%%
%% MEMO: (for myself, FN) shell: %
%% xelatex filebase              %
%% biber filebase                %
%%%%%%%%%%%%%%%%%%%%%%%%%%%%%%%%%%

\documentclass[output=paper,colorlinks,citecolor=brown,
% hidelinks,
% showindex
]{langscibook}

%% Changes to langscibook.cls -> had to comment out: \reserveinserted{18} (FN)

\author{Friedrich Neubarth\affiliation{Austrian Research Institute for Artificial Intelligence (OFAI); University of Vienna}}

\title{If size ever matters, let's compare}

\abstract{Originally intended as a manifest allusion to the title of this book, I selected comparatives in the context
of negative polarity items (NPIs) as the guiding theme of this paper. Although it seems well understood, why weak NPIs
are compatible with the standard of comparison, two other facts have escaped a bit attention: the items in question
always receive an interpretation that is characteristic for universals (even \textit{ever} in English, generally taken
to be a genuine existential NPI); and strong NPIs are totally unacceptable. Trying to establish an analysis of weak
NPIs in terms of a Hamblin-style semantics (not unprecedented in the literature), opens an interesting path and seems
feasible, although many details need to be worked out further. The second issue, infelicitous strong NPIs in
comparatives, can well be aligned to the fact that negation itself must not occur in such a context, since the meaning
of the comparative would be undefined then. Finally, contemplating sub-trigging cases of weak NPIs in terms of
Hamblin-sets opens up further space for such a trait, and presumably offers a better explanation of what actually goes
wrong when weak NPIs are not licensed properly.}

%move the following commands to the "local..." files of the master project when integrating this chapter
% \usepackage{tabularx}
% \usepackage{langsci-basic}
% \usepackage{langsci-optional}
% \usepackage{langsci-gb4e}
%\bibliography{localbibliography}

%% removed by FN
%% \newcommand{\orcid}[1]{}
%%
%% remark by FN: shouldn't it be "\IfFileExists{../localcommands.tex}"?
%% \IfFileExists{./localcommands.tex}{%hack to check whether this is being compiled as part of a collection or standalone
%%   % add all extra packages you need to load to this file
\usepackage{tabularx,multicol}

\usepackage{url}
\urlstyle{same}

\usepackage{listings}
\lstset{basicstyle=\ttfamily,tabsize=2,breaklines=true}

\usepackage{langsci-basic}
\usepackage{langsci-optional}
\usepackage{langsci-lgr}
\usepackage{subcaption}
\usepackage{expex}
\usepackage[linguistics]{forest}
\forestset{qtree edges/.style={for tree={parent anchor=south, child anchor=north}}}
\usepackage{graphbox}
\usepackage{leipzig}
\usepackage{multirow}
\usepackage{needspace}
\usepackage{newunicodechar}
\usepackage{pifont}
\usepackage{qtree}
  \qtreecenterfalse
\usepackage{soul}
\usepackage{tabto}
\usetikzlibrary{calc,decorations,shapes.misc,decorations.pathreplacing,arrows}
\usetikzlibrary{}
\usetikzlibrary{}
\usepackage{tikz-qtree,tikz-qtree-compat}
\usepackage{tree-dvips}
\usepackage{verbatim}

\usepackage{langsci-gb4e}

%%   \newcommand{\orcid}[1]{}


% pesetskycommands.tex
%ExPex stuff

\newcommand{\pexcnn}{\pex[exno={~},exnoformat={X}]}  % pexc with no examplenumber and no increment
\lingset{everygla=\normalfont,{aboveglftskip=0pt}}
\newcommand{\ptxt}[1]{#1\par}
\let\expexgla\gla
\AtBeginDocument{\let\gla\expexgla\gathertags}

%other stuff
\newunicodechar{⟶}{{\symbolfont{⟶}}}
%\newunicodechar{✓}{{\Checkmark}} % checkmark
\newunicodechar{✓}{{\ding{51}}} % checkmark
\newcommand{\gap}{\ \underline{\hspace{10pt}\phantom{x}}\ }
\newcommand{\ix}[1]{\textsubscript{\it{#1}}}  % subscript

\def\gethyperref#1{\hyperlink{#1}{\getref{#1}}}%
\def\getfullhyperref#1{\hyperlink{#1}{\getfullref{#1}}}%

\newcommand{\denote}[1]{⟦{#1}⟧}

%%   %% hyphenation points for line breaks
%% Normally, automatic hyphenation in LaTeX is very good
%% If a word is mis-hyphenated, add it to this file
%%
%% add information to TeX file before \begin{document} with:
%% %% hyphenation points for line breaks
%% Normally, automatic hyphenation in LaTeX is very good
%% If a word is mis-hyphenated, add it to this file
%%
%% add information to TeX file before \begin{document} with:
%% %% hyphenation points for line breaks
%% Normally, automatic hyphenation in LaTeX is very good
%% If a word is mis-hyphenated, add it to this file
%%
%% add information to TeX file before \begin{document} with:
%% \include{localhyphenation}
\hyphenation{
affri-ca-te
affri-ca-tes 
agree-ment
anaph-o-ra
an-ti-caus-a-tive
an-ti-caus-a-tives
an-te-ced-ent
caus-a-tive
Christo-poulos
clas-si-fi-er
Co-lum-bia
com-ple-ment-izer
com-ple-ments
con-tin-u-a-tive
de-di-ca-ted
De-mo-cra-tas
Dor-drecht
du-ra-tive
Ex-folia-tion
ex-tra-gram-mat-ical
fi-nite-ness
ger-und
ger-unds
Gro-ninger
Japan-ese
judg-ment
Judg-ment
Lysk-awa
Ma-khach-ka-la
Ma-rantz
Mat-thew-son
Max-Share
merg-er
Pap-u-an
Per-elts-vaig
post-ver-bal
phe-nom-e-non
pre-dic-tion
Pre-dic-tion
Rich-ards
to-pic
Wil-liam-son
}

\hyphenation{
affri-ca-te
affri-ca-tes 
agree-ment
anaph-o-ra
an-ti-caus-a-tive
an-ti-caus-a-tives
an-te-ced-ent
caus-a-tive
Christo-poulos
clas-si-fi-er
Co-lum-bia
com-ple-ment-izer
com-ple-ments
con-tin-u-a-tive
de-di-ca-ted
De-mo-cra-tas
Dor-drecht
du-ra-tive
Ex-folia-tion
ex-tra-gram-mat-ical
fi-nite-ness
ger-und
ger-unds
Gro-ninger
Japan-ese
judg-ment
Judg-ment
Lysk-awa
Ma-khach-ka-la
Ma-rantz
Mat-thew-son
Max-Share
merg-er
Pap-u-an
Per-elts-vaig
post-ver-bal
phe-nom-e-non
pre-dic-tion
Pre-dic-tion
Rich-ards
to-pic
Wil-liam-son
}

\hyphenation{
affri-ca-te
affri-ca-tes 
agree-ment
anaph-o-ra
an-ti-caus-a-tive
an-ti-caus-a-tives
an-te-ced-ent
caus-a-tive
Christo-poulos
clas-si-fi-er
Co-lum-bia
com-ple-ment-izer
com-ple-ments
con-tin-u-a-tive
de-di-ca-ted
De-mo-cra-tas
Dor-drecht
du-ra-tive
Ex-folia-tion
ex-tra-gram-mat-ical
fi-nite-ness
ger-und
ger-unds
Gro-ninger
Japan-ese
judg-ment
Judg-ment
Lysk-awa
Ma-khach-ka-la
Ma-rantz
Mat-thew-son
Max-Share
merg-er
Pap-u-an
Per-elts-vaig
post-ver-bal
phe-nom-e-non
pre-dic-tion
Pre-dic-tion
Rich-ards
to-pic
Wil-liam-son
}

%%   \bibliography{localbibliography}
%%   \togglepaper[23]
%% }{}

\begin{document}

\maketitle

%%%%%%%%%%%%%%%%%%%%%%%%%%%%%%%%%%%%%%%%%%%%%%%%%%%%%%%%%%%%%%%%%%%%%%%%%%%%%%%%
% BEGIN TEXT                                                                   %
%%%%%%%%%%%%%%%%%%%%%%%%%%%%%%%%%%%%%%%%%%%%%%%%%%%%%%%%%%%%%%%%%%%%%%%%%%%%%%%%

\section{Introduction}

Initially, it seemed a bit tricky to find anything that would justify the participation in a book project dedicated to
the notion of size in linguistic theory, when having basically focused on polarity items (PI) in linguistic work. But
then, the idea occurred to me that I could refer to size as a dimension in the object language, steering the discussion
towards comparatives, which are quite an interesting environment in the context of negative polarity items (NPI).

Still, I searched for more to say rather than replicating my not all too seminal analysis of weak NPIs with universal
force in the standard of comparison. Which in itself is puzzling and provides us good insights about the nature of
NPIs. Fortunately, I stumbled over the long-known fact that comparatives, while being provident licensors for weak NPIs,
seem to be in conflict with negation, on a par with strong NPIs. But, no rule without exception - and these exception
bears some analogy to sub-trigging cases where weak NPIs apparently live well without overt or covert licensors, the
only thing that has to obtain is contingency. Only within a greater context of phenomena it becomes really worthwhile
to search for better answers to long-pending questions.

However, the issue at the core of all this is actually the size of the set of potential referents. That size is at
stake was formulated in \citet{kadmonlandmann1993} who reported the effect of widening induced by weak NPIs (plus
strengthening). Both terms were re-engineered by Chierchia on various occasions \citet{chierchia2004,chierchia2013}.
Alternatively, \citet{krifka1995} follows a different trait. On his account, weak NPIs such as \textit{any} in English
denote the entirety of entities that comply with the properties defined by the noun (phrase). Super-size -- without limits!
This particular property can be made responsible for the particular behavior of so-called weak NPIs. Size matters,
indeed.

\section{Some remarks on NPIs}

To my knowledge, the term NPI was coined by \citet{klima1964} and referred to weak NPI \textit{any} that was related to
its alleged positive polarity item (PPI) counterpart \textit{some} by a set of transformational rules. That this
relation was not at all warranted has been shown by \citet{lakoffrob1969}: Questions are a grammatical context that
licenses both items equally well (e.g. ``who wants \{any/some\} beans?''). The only difference that can be detected is
that the PPI \textit{some} triggers an existential presupposition, whereas the NPI \textit{any} definitely does not. It
can be interpreted neutral (in the well known sense that it renders total indifference towards which beans) or with a
negative bias when focussed, expressing the expectation of a negative answer. It is important to note that this bias is not
obligatory. Regarding the PPI \textit{some}, I still contend that it is an indefinite carrying along a presupposition
of existence (see \citet{neubarth2006}). Therefore its resistance to be in the scope of negation, unless negation
is cancelled out. Definitely, it is not a counterpart to NPI \textit{any} in any way.

The story is well known, but I need to rehearse a few details, since they may turn out crucial for the analysis I want
to establish later. \citet{ladusaw1979} was the first to note that entailment properties of the semantic context play a
crucial role, while a few years later \citet{linebarger1987} raised the claim that (syntactic) negation plus pragmatic
factors akin to the bias mentioned before should be taken responsible for licensing NPIs. What she overlooked, though,
is that NPIs do not form a uniform class. Mostly, weak NPIs such as \textit{any} were taken into consideration, but
analyses so far mixed up those with the other class (strong NPIs such as \textit{a single N, budge an inch} etc.). From
an empirical perspective it is clear that strong NPIs create a bias in many cases, while weak NPIs may well have a
neutral interpretation also in environments that do not involve negation. It was \citet{heim1984} who noticed this
difference for the first time, and \citet{zwarts1990, zwarts1998} came up with an analysis in terms of
downward-entailing (DE) vs. anti-additive environments. His work was also responsible for establishing the distinction
between strong and weak NPIs, now generally used in the literature.

Glossing over three decades of investigation, ideas, and disputes (I will refer to those whenever it is necessary), I
will try to lay out what I believe to be the relevant properties of these two types of NPIs.

\subsection{Strong NPIs}

NPIs of this class always have a quantified NP that can be interpreted as a minimal quantity (e.g.,
\textit{so much as a single} N), and most often they are equipped with (at least) one
focus-attracting particle (\textit{even a single} N). A few of them are indeed idiomatic predicates
(e.g., \textit{budge an inch}). Consider the example from \citet[104]{heim1984}:

\protectedex{
  \ea \label{neubarthex:everyrestaurant}
  \glt Every restaurant that charges so much as a dime for iceberg lettuce \{ought to be closed down / \textsuperscript{??}actually has four stars in the
       handbook\}.
  \z
}

The restriction of universals is clearly a DE environment. By virtue of that, all propositions with members on a scale
consisting of alternatives to the minimal quantity expressed by \textit{a dime} will be entailed by the sentence given.
These may be higher numbers than just one, or more valuable monetary units. In other words, the proposition with the
minimal quantity yields the strongest assertion. If the context for the NPI were not DE, the assertion would be the
weakest possible, and -- frame it as you wish -- it is clear that such a sentence would be unacceptable. Thus, having
established the need for a DE context (as proposed by \citet{ladusaw1979} and others), the question is still
pending why the two examples (merged into one) are different. The second version, where the universal subject DP (or rather QP) is in
a non-modal, indicative environment is clearly unacceptable.

The rationale behind is that it is not a minimal quantity, but actually interpreted as a minimizer. It cannot be a real
quantity (albeit it has the linguistic form of one).\footnote{Manfred Krifka (p.c.) once pointed out to me that this
property could well be compared to the epsilon in infinitesimal calculus: an arbitrarily small positive quantity. His
definition in \cite{krifka1995} involves exhaustivity.} The only way to achieve this effect is to exclude a situation
where the extension of the DP in the evaluation world is not the empty set. Negation provides the right context, weak
DE quantifiers such as \textit{few} clearly do not, in fact they explicitly assert that the extension has ``a few''
members. When dealing with the restriction of universal quantifiers or antecedents of conditionals it is not clear
whether there is an extension in the real world. As Heim already noticed, adding an appropriate modal to the sentence
already does the job -- prohibiting an inference to the real world.

While the acceptable version of (\ref{neubarthex:everyrestaurant}) could be paraphrased as some kind of threat, the following
example clearly cannot, indicating that it is just the mere ban on extension in the evaluation world induced by the
minimizer, rather than some pragmatic devices, as suggested by \citet{linebarger1987}. Consider the following example
in German:

\protectedex{
  \ea
  \gll \textit{Wenn} \textit{du} \textit{auch nur} \textit{ein} \textit{einziges} \textit{St\"uck} \textit{von} \textit{dieser} \textit{Torte} \textit{kostest,} {\{\textit{wirst} /} \textit{w\"urdest}\} \textit{du} \textit{sehen,} \textit{dass} \textit{sie} \textit{irgendwie} \textit{doch} \textit{gut} \textit{schmeckt.}\\
       if you even a single piece of this tart try, {\{will-\textsc{ind} /} will-\textsc{cond}\} you see, that it somehow nevertheless good tastes.\\
  \glt `If you tried even a single piece of this tart, you would see that nevertheless it tastes somehow good.'
  \z
}

To be on the safe side I used the hypothetical conditional for the English translation. In German, the use of the
subjunctive is not obligatory. But notice that the conditional in both languages is not counter-factual. What is more
interesting is first that we have an NPI with a focus-attracting particle, and second that in German what is expressed
by \textit{even} in English actually involves two particles: additive \textit{auch} (`also') and exclusive \textit{nur}
(`only').\footnote{The same is true for Italian (see \citet{guerzoni03}).} In English, the role of the particle
\textit{even} is marginal (otherwise we would not find strong NPIs without it). Basically, it fosters the scalar
reading by presupposing a scale of likelyhood, where the element in focus is at the bottom of that scale (see
\citet{leehorn1994}). This likelyhood scale matches the scale of entailments in DE environments, a crucial
circumstance. The German (or Italian) case is striking, though, since neither of the particles is scalar in nature.

Starting with the `inner' particle \textit{nur} (`only'), its semantics is that it presupposes the truth of the
proposition applied to the element in focus and asserts that for all alternatives (ordered or not) the proposition will
be false (\citet{krifka1998a,wagner2005}). When in the scope of negation, the implicit universal operator (over
alternatives) is negated and the interpretation of exclusiveness is explicitly denied. In our case, as part of an NPI,
\textit{nur} presupposes the truth of the minimal element and asserts the falsity of all other elements. This is
definitely not what we want for \textit{nur} as part of an NPI. Remember, that the scalar reading is induced by virtue
of the minimal element being interpreted as a minimizer alone, and the particles just reinforce that interpretation.

The second, additive particle has almost contrary properties: \textit{auch} ``expresses that the
predication holds for at least one alternative of the expression in focus''
\citep[111]{krifka1998a}. So, in fact it is the additive particle that reinforces the scalar
interpretation. But what are the alternatives? Following an idea proposed by Manfred Krifka (p.c.,
at SemNet 2007 in Berlin), the alternatives are expressions of the form ``\textit{nur} X'' where X
stands for the complex of a (numerical) quantifier Q and its restrictor, the NP part, whereas Q can
have any value different from one (the minimal value). Due to the mandatory DE-context, all these
alternatives are entailed by the minimizer, as before, whereas likelyhood coincides with entailment.

The interplay of the two particles is structured as follows: the higher particle \textit{auch} enforces presupposed
alternatives (that need to be true), but it shields the expression headed by the lower particle from the effect that
negation (if there is) would target the universal over its alternatives.\footnote{I am not sure if this could be regarded as a
case of intervention as observed with quantifying expressions.} Rather, the lower particle \textit{nur}
establishes a scheme that defines the alternatives of the higher particle, i.e. ``\textit{nur} X'', where alternatives
vary over X (or better, the quantifier contained in it). Intuitively, it is clear what happens: the meaning of
\textit{nur} only applies within its domain, the focus or one of the alternatives of \textit{auch}. Outside, on a
higher structural level of semantic interpretation, neither the presupposition, nor the universal over alternatives are
visible. Rather, a higher scoping negation would target the indefinite within. I admit, this is a mere sketch, more
work would be needed to specify out the details in a cleaner manner.

Krifka's \citeyearpar{krifka1995} proposal for strong NPIs involves an emphatic assertion operator. These sketchy ideas are not
only well compatible with his approach, they actually derive from it. The scalar nature of strong NPIs is sometimes
triggered by lexical aspects (e.g., ``a dime''), in other cases by focus attracting particles, where focus makes
(quantificational) alternatives `visible'. Krifka is very cautious about identifying the causality of the particular
behavior of strong NPIs. Clearly, they convey an ``extreme'' meaning w.r.t. their position in the induced scale. He also
notes that Zwart*s \citeyear{zwarts1990} notion of anti-additiveness is too strict, given that examples with ``extreme'' items
can license strong NPIs without fulfilling the requirement of being anti-additive in a strict sense (e.g., \textit{``Hadly ANYONE
lifted a FINGER to help me.'' \citet{krifka1995}}). Needless to say that such examples are also a challenge for the apparently not always applicable claim for a ban on
extensional instantiation.\footnote{As one reviewer pointed out, this situation seems close to what \citet{giannak2007} identified
as the property of an antiveridical operator -- to prevent extension in a world of evaluation. It is not fully
clear to me, if her notion of antiveridicality does exactly the job. Furthermore it is doubtful whether we need to categorize the context in such a way, rather than
focusing on the semantic properties of the items in question itself, and their interpretation. Clearly, Krifka*s notion of
``extremeness'' needs further attention.}

Another issue with licensing of strong NPIs in the standard of comparatives is, again, the
status of its context. It can be shown that these contexts are a least Strawson anti-additive, which should render them licensed, according
to standard assumptions. They are not licensed, and this is
one more indication that we need to turn our attention more to the semantic properties of particular items rather than
searching for the right definitions of the contexts that would license a particular item.\footnote{\citet{gajewski2010} remarks
that Strawson entailment relations are not at issue with licensing NPIs. However, he discusses only superlatives, leaving open the
question whether his findings can be transferred to comparatives as well. I guess so.}

To sum up, strong NPIs require a DE context and by virtue of their interpretation as minimizers they also require that
the extension of the expression w.r.t. the evaluation world is the empty set. In this formulation, it emulates
more or less \citet{zwarts1998} characterization that strong NPIs demand for an anti-additive context, but it finds better remedy
for the observations in \citet{heim1984} that would remain unaccounted for. This will be essential for an explanation
why strong NPIs are not licensed in the context of comparatives.

\subsection{Weak NPIs}

This type of NPIs is of an entirely different nature. This can be illustrated by the mere fact that often NPI-\textit{any} is discerned from a
free-choice item (FCI) \textit{any}. \citet{horn2000a} undertakes
a comprehensive survey about which authors have lent themselves to a univocal existentialist or universalist approach, and
which have adopted lexical ambiguity. While I would object to the last option on conceptual grounds, the data indeed
oscillates between a quasi-existential and a quasi-universal interpretation. In \citet{neubarth2017} I have expressed
my unhappiness about this alleged bifurcation of interpretations and shown that in the context of comparatives the NPI
\textit{ever}, generally taken for existential, also receives a quasi-universal interpretation. My conclusion then was
that the distinction between existential and universal should not be applied at all with weak NPIs.

Quoting from there, the most sensible meaning of weak NPIs is ``that they actually denote the set of all possible
referents that fulfil the properties denoted by the noun phrase (including cardinally modified pluralities, such as
{\em any two X}), or, in case of {\em ever}, the set of all relevant (accessible) situations/times.'' This is
reminiscent of a Hamblin-style semantics \citep{hamblin1973} and would explain, why we never get effects of existential
closure, or in the sense of \cite{reinhart1997} the application of a choice function that would determine the reference
of an (indefinite) nominal expression.

What I still take to be essential from there is that under the view that lexical items have just one meaning (save
polysemy, rarely found with grammatical lexemes, if ever) it cannot be a `special' property of these items to denote
either existiential, quasi-universal or free-choice meanings, but rather the other way round. Elaborating a bit
further, these items just block what normally happens when we encounter an non-definite nominal. Therefore its
interpretation can simulate both, a universal or an existential interpretation. A classic example are conditionals --
in the case at hand with two different continuations:\footnote{One reviewer has commented that this example contains
the modal ``can'', thus allowing for an ambiguity between a free-choice and an existential reading. This is correct,
and exactly the case: the modal enables both readings. However, what the example really shows is that the
interpretation switches between an existential or a free choice interpretation, sensitive to the context which
determines direction of a scale of likelyhood (or expectation). I would argue that the meaning of {\em any} does not
change, but its interpretation does, accordingly to context.}

% \protectedex{
  \begin{exe}
    \ex\label{neubarthConditional} If she can solve any of these three problems
    \begin{xlist}
      \ex[]{she must be a genius.}
      \ex[]{she has good chances to pass the test.}
    \end{xlist}
  \end{exe}
% }

The partitive is on purpose here, it excludes an interpretation where widening takes place, problematic for
\citet{kadmonlandmann1993}, but presumably amenable. In (\ref{neubarthConditional}a) we get a quasi-universal interpretation,
however, mediated through the FCI \textit{any}, in the sense of `no matter which'), in (\ref{neubarthConditional}b) there is a
quasi-existential interpretation at stake, still affected by the `no matter which' premise, but it seems enough to
solve one problem in order to fulfill the antecedent of the conditional. Under a closer look, we see that the two uses
of {\em any} are tied to different scales of expectation, targeting likelyhood. In the first variant the continuation
enforces a low expectation to solve all three problems, where the reverse is true for the second variant. In the first
variant, the full set is the domain of reference, whereas in the second variant there is indifference about the choice,
but any choice fulfills the requirements, cf. \citet{dayal2004}. Under closer scrutiny, we see that it is barely the
responsibility of the weak NPI to trigger those two interpretations. Rather, the context determines whether one is
sufficient, or `any of these three problems' is interpreted as covering the entire set.

Setting those things aside for a bit, let us move on to investigate the perennial question what makes weak NPIs infelicitous in
non-negative, declarative contexts? \citet{krifka1995}, to my opinion, has provided the most intuitive definition of weak NPIs:
an expression of the form {\em any X} denotes an entity out of the set of all entities that fulfill the property X, but
deliberately in the most unspecific way. Meaning, that there deliberately is no contextual specification to be revealed
or implied. Logically, a proposition $p$ containing such an expression is weaker than any other proposition containing
a more specific, alternative expression. Krifka proposes a pragmatic principle of scalar assertion, where scalar
refers to ordered strength. His argument is that for every potential alternative proposition $p'$ being stronger than
$p$, this proposition must not be true (a line of reasoning that has been taken up again by \citealt{chierchia2004}).

In the case that stronger propositions can be true, there is no direct contradiction, however, the assertion becomes
undefined. In a DE context, where the direction of entailment is reversed and the most general expressions yield the
strongest propositions, all propositions containing alternatives become weaker than $p$. Following Krifka, we can
safely deduce that DE contexts provide an environment for weak NPIs where they can contribute to interpretable
(=defined) assertions. When a weak NPI such as \textit{any} has focus, the alternatives become explicitly visible and
propositions $p'$ containing them, need to be false by pragmatic principles. Yet again, but even more so, a felicitous
interpretation depends on scale reversal, in order to provide a stronger (or the strongest) statement while widening
the domain of reference to its maximum. That far, it is more or less compatible with \citet{kadmonlandmann1993}, who,
admittedly, do not refer to DE as a condition for \textit{any}, and do not give an independent explanation for the
infelicity of certain contexts, but rather stress the effect of widening the domain of reference.

\citet{dayal1998}, not without justification, goes as far as to propose a universal operator that leads to a
presupposition failure when occurring in non-subtrigged epistemic contexts. In order to remedy quasi-existential
interpretations, but also to ensure that FCI {\em any} is licensed by a possibility operator, and not by necessity
operators (without further modification). This requirement of indeterminacy, ``{\em as a grammatical constraint against
the extension of the relevant property (the intersection of the nominal and the verbal properties) being the same in
every accessible world}'' \citep[237]{dayal2009} was refurbished as `fluctuation', and  in \citet{dayal2013}, she
formulates a 'Viability Constraint on Alternatives', aligning her analysis to Chierchia's \citeyearpar{chierchia2011}
account of NPIs. It provides a more independent grounding within a semantics dealing with alternatives. The connective
idea behind is that FCI {\em any}, while being an indefinite, hooks up to a universal operator. In her most recent
account based on `Viability', this universal operator arises as a (FCI) implicature that is enforced by negating all
exhaustified sub-domain alternatives.

The discussion revolves about the question how to achieve the Janus-faced interpretation and in the case of a
quasi-existential interpretation how to ensure the property of indeterminacy. Other accounts on NPIs make more direct use
of a semantics based on the work of \citet{hamblin1973}
(see \citet{ramchand1997,kratzershimo2002,kratzer2005,novelromero2009} among others, especially substantial elaborations
on the nature of free-choice effects in \citet{fox2007,chierchia2013}). The meaning of an indefinite is
not quantificational per se, but actually the set of alternatives (contextually available referents). This set percolates up,
potentially to the propositional level, but may also be closed off by an appropriate operator, actually the first one
occurring during the compositional procedure. This still does not explain our apparent ambiguity, but at least it
explains why NPIs keep licensed in cases where a higher operator might reverse the scale again (e.g., double negation).

All this calls for a formal elaboration I am not in the position to deliver here. But for the purposes of what
follows, it may be sufficient to recapitulate the following: i. indefinite expressions are analyzed as Hamblin-sets,
ii. weak NPIs (also in their guise as free-choice items) generally resist whatever means of `existential closure' -- that sets them apart from common indefinite
expressions, iii. the quasi-existential interpretation comes about when existential closure is `imported' from
somewhere else, while keeping truth conditions on the whole set intact. iv. the quasi-universal reading receives a
natural explanation since it reflects entire access to the members of a Hamblin-set.\footnote{Reacting on one
reviewer's comment that `Hamblin sets [are] to be [taken] disjunctively combined, i.e. resulting in existential
meanings', I would object that this is not at the core of a (quasi-)Hamblin semantics. Whether an indefinite receives
an existential meaning depends on the context (and most often the context provides existential closure). Free-choice
has always been a problem since it is neither genuinely universal, nor existential in the common sense, but rather
quasi-existential (``take whichever you choose'') or quasi-universal (``you may choose out of the whole set''),
reminiscent of Morgan's Law that allows for replacing an existential with a universal operator while swapping the scope
of negation.}

A bit more needs to be said about items ii.--iii.: Definitely, \textit{any}+NP expressions, and their kins, are
different from plain indefinite expressions, and while they might share a few intersections with those for example in
generically interpreted contexts \citep[cf.]{kadmonlandmann1993}, they cannot be subject to existential closure, or
binding by a choice-function, whatever you take -- the Hamblin-set prevails. Unfortunately, this remains a mere stipulation,
not deduced by other criteria. Regarding the quasi-existential interpretation, I reckon that it comes about either when
we have negation involved, but this is a bit of an illusion: de facto the interpretation is non-existential. Or with downward entailing quantifiers
(e.g., ``Few voters of the president have read any book.''), where the quantifier \textit{few} provides a context where
the availability of the whole set is not violating pragmatic principles of assertion in the sense of
\citet{krifka1995}. Nevertheless, it is interesting to observe that in German the correspondent to the given example
involves the same construction that has been discussed with strong NPIs:

\protectedex{
  \ea
  \gll \textit{Wenige W\"ahler} \textit{haben} \textit{auch nur irgendein} \textit{Buch} \textit{gelesen.}\\
       {few voters} have any book read.\\
  \glt `Few voters have read any book.'
  \z
}

Recall that the \textit{nur} particle confines the reference to the item in scope, by its presupposition, while negating
the truth of the alternatives. On the other hand, the higher additive particle \textit{auch} reinforces alternatives and shield the
lower particle from percolating its presupposition (which needs to be treated in a more dynamic way, as we see), but
also shields the assertion from reversing truth value under negation. My tentative formulation was that this
encapsulation takes place by enacting a higher order treatment on alternatives, where the \textit{nur} (`only') DP
expression serves as a scheme that is not evaluated outside the scope of the higher particle. As
\citet{kratzershimo2002} and \citet{kratzer2005} point out, the contribution of \textit{irgendein} is to make the
alternatives being kept available in a way. What is striking is that before, when dealing with quantificational
minimizers, which yield strong NPIs, there was no way to escape the ban on manifest extensions, while here it seems
as if we have found a (partial) correspondent of English \textit{any}, well suited for a quasi-existential
interpretation.

This is the point where we have to perform some sort of looping. When the expression \textit{irgendein Buch} is
introduced, it refers to a potentially infinite set of books. In German, the particle \textit{nur} selects one, most
unspecific -- and that is the loop -- by virtue of being most unspecific. Then we are with \citet{krifka1995} and his
reasoning. DE is mandatory, of course, but it is warranted. But why not a ban on non-empty extensions? I would contend
that this is exactly because there is no quantitative entailment. Any individual assignment of `few voters' to books they
might have read does not contradict the whole. In other words, the Hamblin-set gets evaluated at the level of
interpretation of the quantifier `few' that itself is exclusive towards an unspecified majority. Once this evaluation
has taken place, the Hamblin-set will not be accessible at the level of proposition. Again, this would need a formal elaboration, but
what I aim at here is just to push the conceptual idea of utilizing Hamblin-sets in a more general way, admittedly in a more
or less naive manner. Let us move on to the main topic or this paper, NPIs in the standard of comparatives.

\section{NPIs in the standard of comparatives}

Despite of the complexity that comparatives offer, with respect to NPIs the situation is surprisingly clear. Weak NPIs
are absolutely natural in the standard of comparison (and not licensed outside, of course), strong NPIs are strictly
ungrammatical.

\subsection{Weak NPIs and comparatives}

Let us leave out the latter for a moment and start with weak NPIs. Actually, as of what has been said before, it is not
a miracle that they live so well there: the standard of comparison is a DE environment. What is puzzling is that the
interpretation weak NPIs receive is a (quasi-)universal one, as noted by \citet{schwarzschwilk2002}. When representing
the denotation of indefinites as Hamblin-sets that resist existential closure, they just live up to where the standard
of comparison is evaluated. I want to discuss
this effect with a classical accounts on comparatives, notwithstanding that there is much more to be said on
comparatives per se.

The universal interpretation weak NPis always have in the standard of comparison needs to be highlighted -- even
\textit{ever} that is normally taken to be confined to (quasi-)existential interpretations receives such an interpretation. Building on those findings,
I have tried in \citet{neubarth2017} to defend a position where the distinction between existential and universal
interpretations should be abandoned for these kinds of NPIs, though not providing all too viable alternatives.
For ease of demonstration, I will refer to von Stechow's \citeyearpar{stechow1984} analysis of
comparatives.\footnote{This approach has some well-known difficulties, particularly in connection with universal quantifiers. Various authors have
tackled this problem, coming up with solutions that are closer or more distant from von Stechow*s foundational proposal.
See for example \citet{schwarzschwilk2002,heim2006,beck2010,fleisher2016}.}
That weak NPIs receive a universal interpretation has already been noted in that paper, and later in
\citet{schwarzschwilk2002}, albeit just in a footnote. Von Stechow's formalization rests on the assumption that the standard
of comparison (the \textit{than}-complement) determines a property of degrees, one can abstract over, rendering the
whole a nominal with scope, thus enabling, but also reinforcing raising. One of the most appealing features of von
Stechow's analysis is that it treats semantics and syntax on a par. The meaning of the comparative complement can be
represented as ``the(Max(P))'' \citep[55]{stechow1984}, where `the' stands for the Russellian definite description
operator, and `Max(P)' is maximization over degrees that is defined in such a way that it is the property being true of
any degree $d$ in a word $w$, given that there is no other degree $d' > d$ that would be true in this world as well.

Various analyses are possible regarding weak NPIs. Pushing the Hamblin-set idea, it appears quite plausible that the maximization function
of the comparative would be satisfied to evaluate over an indefinite that offers a (non-closed) Hamblin-set. This would yield a
nice explanation for the quasi-universal interpretation: maximization
evaluates the whole set indiscriminately, so its character as a set prevails. Later accounts on comparatives, i.e.
\citet{stechow1996,heim2000} do not depart from the core insight that there is a function that goes over all
alternative degrees ensuring that the conditions expressed by the comparative are met. And needless to say that all
having been said extends to the equative as well, since its meaning also rests on maximization.

Notice that it might not be the DE property of comparatives that `licenses' weak NPIs of the \textit{any X} type. If this
were indeed the case, the hybrid NPI in German with focus-attracting particle but no quantifier should equally well be
licensed (as in other DE contexts). However, the sentence, while not being fully unacceptable, is considerably odd:

\protectedex{
\begin{exe}
  \ex[??]{
    \gll \textit{Gustav ist gr\"o{\ss}er als} \textit{auch nur} \textit{irgendeiner} \textit{von seinen Kollegen.}\\
         {Gustav is taller than} {even} {anyone} {of his colleagues}\\
    \trans `Gustav is taller than ANY of his colleagues.'
  }
\end{exe}
}

Although the context is DE, it seems that the scalar nature of the NPI in question gets in conflict with maximization
that expects an unordered, fully accessible set of alternatives to apply upon. Weak NPIs of the \textit{any} type, by virtue of
representing Hamblin-sets on the other hand provide ideal conditions for maximization.

\subsection{Strong NPIs, negation and comparatives}

The fact that negation or negative expressions in general are not possible in the standard of comparison is well known
at least since \citet{lees1961}, among others, \citet{ross1980} and \citet{stechow1984} commented on it, but I will
mainly refer to \citet{lechner2002}, since it extends the data in a substantial way, see below.

Employing von Stechow's analysis for comparatives once more yields a straightforward explanation. Alongside Russell he
remarks that it is impossible to apply the definite description operator onto an empty set, since the definite
description simply does not denote then. It is crucial to bear in mind that it is not the maximization function that would create
problem here. Also, von Stechow does not classify such sentences as ungrammatical, but rather as `extremely odd'.
Things being rather clear thus far, the question arises, why the standard of comparison is also adverse towards DE
quantifiers. Consider the following examples:

% \protectedex{
  \begin{exe}
    \ex
    \begin{xlist}
      \ex[??]{Gustav is taller than few of his colleagues.}
      \ex[]{Gustav is smaller than many of his colleagues.}
      \ex[]{Few of his colleagues are smaller than Gustav.}
    \end{xlist}
  \end{exe}
% }

We are dealing with a linear scale of size, here. All three sentences would mean that the size of Gustav is in the
lower range of all sizes abstracted over his colleagues and him. This can be stated explicit, as in the (b.) example,
or implicit, as in (c.) that asserts that the number of colleagues with a size below that of Gustav is small. Applying
a proportional reading on \textit{few} gives us the correct result. But (a.) is odd. I have no better explanation for
this other than that a quantifier such as \textit{few} lends itself to a scalar implicature comprising the empty set
(`\textit{few if not none'}).

Interestingly, \citet[12]{lechner2002} comes up with a case where negation can occur with a comparative. This is what
he calls `parallel comparatives':

\begin{exe}
  \ex[]{Mary read more books than she didn't read.}
\end{exe}

He further notes that this effect is only possible with count nouns, not with predicative, attributive comparatives,
and not with mass terms, stating that ``it seems as if a bi-partition can be established only if the comparison
relation operates on degrees that keep track of cardinality (as in \textit{d-many books})''. I reckon, this is already
an explanation: only when the set of entities yielding degrees to abstract over is a contingent complement to another
set of a larger set of entities (i.e. \textit{books she didn't read} vs. \textit{books she read}) is it possible to
fulfil the requirement that the maximized set of degrees can denote.

Now, with strong NPIs we might wonder why they are impossible in a context that can be shown to be DE. Actually,
comparatives are a good test-case to discern strong from weak NPIs, which are fine in comparatives. But it is not the
mere lack of an anti-additive operator, as Zwarts would have stated, but the fact that comparatives and strong NPIs
bear contradicting conditions. While comparatives, as we have seen, need to make sure that the set of extensions w.r.t.
to entities with a property that yields degrees must not be empty, strong NPIs demand the opposite, otherwise their
own meaning leads to a contradiction.

\section{A note on contingency: sub-trigging}

Albeit by mere analogy, but the last example reminded me of one of the most interesting puzzles in connection with
weak-NPIs. As already noticed in \citet{legrand1975}, weak NPIs may show up in simple declaratives sentences given that
their restriction is confined in an appropriate way, see \citet{dayal1998} for an extensive discussion and
analysis.\footnote{It has to be noted that I use the term `weak NPI' for \textit{any} in a generalized way, building upon the assumption
that there is no lexical ambiguity involved. Other authors, including \citet{dayal1998}, systematically
discern between (weak) NPIs and free-choice items, with varying implications.} In fact, it is in this paper where she defines
the essential condition on sub-trigging: the quantified expression must be restricted essentially and propositionaly in such a way that it provides its own situation variable that,
however, must be able to extend into the situation variable of the whole sentence. Consider the revealing contrast in
the following pair \citep{dayal1998}:

\begin{exe}
  \ex\label{neubarthex:sub-trigging}
  \begin{xlist}
    \ex[]{Bill offered Mary everything/*anything he had cooked for dinner.}
    \ex[]{Those days Bill offered Mary everything/anything he cooked.}
  \end{xlist}
\end{exe}

While (\ref{neubarthex:sub-trigging}b.) fulfills the conditions of contingency, the first sentence does not,
since there was only one cooking event/situation that would not be contingent on each individual offering of the
products of Bill's kitchen. What remains to be answered is why non-contingently restricted simple declarative
statements are unacceptable, and why contingent restriction provides remedy.

Recall that Krifka's explanation rested on the assumption that weak NPIs are scalar in terms of specificity. Scalar
assertion in non-DE contexts is not possible on pragmatic grounds, since by scalar implicature every stronger
proposition (applying the scheme to more specific alternatives) needs to be false, but by virtue of the weak NPI
comprising the whole set of entities with a given property (e.g., being a `thing') this is contradicted in non-DE
contexts. Qed, but does it, or how does it carry over to sub-trigging?

What we observe is that weak NPIs denote Hamblin-sets that cannot be turned into a
referential expression by existential closure (or, under an alternative account by applying a choice-function).
However, these Hamblin-sets have to be `tamed' in some way or other. If not, the only interpretation available would be
as an expression that marks a minimum on a scale of specificity, yielding the weakest possible statement in non-DE
contexts, hence Krifka's reasoning must apply. In those contexts where sub-trigging obtains, each member of the
Hamblin-set is individually bound to an event/situation that matches up the event/situation set of the main clause,
hence externally defined. Notice that while the set still remains the same and projects, it is strictly confined (and
also defined) by the set of events/situations of the given main clause. In such a case, but perhaps also in other
cases, the scalar reasoning cannot be applied anymore, and the weak NPI receives a sensible interpretation (the
assertion is meaningful).

\section{Conclusion}

While the beginning of this paper more or less reflects historical facts to lay out the grounds for a critical discussion of the nature of NPIs,
I already tried to indicate where the journey should go: an attempt to solve some open issues in the investigation of
(weak) NPIs in terms of a Hamblin-style semantics. The most intriguing and revealing case here is the behavior of NPIs
in the standard of comparison. Also a bit abusing comparatives to provide the missing link between the work presented
here and the general theme of this book. I have tried to show that an analysis in terms of Hamblin-sets would not only work
well within standard semantic analysis of comparatives (taking \citet{stechow1984} as a point of reference), but
also give a natural explanation to the fact that the interpretation of weak NPIs, including \textit{ever}, is always
(quasi-)universal within the standard of comparatives. On the other side, strong NPIs are not felicitous there, at all.
The reasoning behind is that comparatives and strong NPIs have contradictory conditions regarding the set
of extensions of entities one can abstract the relevant degrees from. While the former require it not to be empty,
strong NPIs demand it to be the empty set. Finally, I speculated about sub-trigging cases which by definition require
contingency between the restriction of the weak NPI and the event/situation frame of the main clause. My tacit proposal
is to make contingency responsible for the blocking of a scalar reading of the NPI, just as in comparatives.

\section*{Abbreviations}

\section*{Acknowledgements}

Contemplating my (long ago) linguistic experiences, I am
very much indebted to Irene Heim, Manfred Krifka and Arnim von Stechow for mind-breaking discussions,
Edwin Williams for summarizing things I did not see as straightforward before and asking the right, difficult
questions, and, last but not least, Martin Prinzhorn who, then as a teacher, was responsible for keeping me up on that
theme. These mentions are only representative for many more I am indebted to express my gratitude to -- for help,
friendship and ideas. Two anonymous reviewers have not only raised my confidence, but, most importantly, raised a few
critical issues that helped me to improve the paper at least into the state where it is now. Also, I need to cordially thank the
editors of this book for inviting me to this project -- making
me feel a linguist again, a bit stressed, barely satisfied, but contented in a certain way.


%%%%%%%%%%%%%%%%%%%%%%%%%%%%%%%%%%%%%%%%%%%%%%%%%%%%%%%%%%%%%%%%%%%%%%%%%%%%%%%%
% END TEXT                                                                     %
%%%%%%%%%%%%%%%%%%%%%%%%%%%%%%%%%%%%%%%%%%%%%%%%%%%%%%%%%%%%%%%%%%%%%%%%%%%%%%%%

% \section*{Abbreviations}
% \begin{tabularx}{.45\textwidth}{lQ}
% ... & \\
% ... & \\
% \end{tabularx}
% \begin{tabularx}{.45\textwidth}{lQ}
% ... & \\
% ... & \\
% \end{tabularx}

\printbibliography[heading=subbibliography,notkeyword=this]

\end{document}
