\documentclass[output=paper]{langscibook}
\author{Laura Kalin\affiliation{Princeton University}}


\title[When size matters in infix allomorphy]
      {When size matters in infix allomorphy: A unique window into the morphology-phonology interface}
\abstract{This paper presents a case study of morphophonology in Nancowry, a dialect of Nicobarese (Austroasiatic; Mon-Khmer; \citealt{Rad81}). In Nancowry, there are several affixal morphemes with exponents that are distributed based on the {prosodic size} of the stem they combine with, and some of these exponents are infixal, appearing in positions where they create opacity. I show that Nancowry provides evidence for (i) the bottom-up cyclicity of exponent choice, infixation, and prosodification, (ii) the serial ordering of these processes within each cycle, and (iii) the largely arbitrary (non-optimizing) nature of exponent choice and infixation. The findings point to a separation of morphology from phonology (in line with, e.g.,  \citealt{HalleMarantz93,HalleMarantz94,Trommer01,Paster06,Yu07,Embick10,BS12,Pak16,Dawson17,Kalin20,Rolle20,Stanton20}), and are consistent with the results from investigating interactions between allomorphy and infixation in a sample of 40 languages \citep{KalinIP}.}


\begin{document}
\SetupAffiliations{mark style=none}
\maketitle
\section{Introduction}

%FOR FINAL VERSION: CHECK THE ENGMAS IN EXAMPLES!!

While there have been a number of surveys and discussions of infixation in the world's languages (\citealt{Mor77,Mor00,Ultan75,Yu07,Stek12,Blevins14}; a.o.) and in-depth case studies on infixation in particular languages (\citealt{HM91,Blevins99,Harizanov17,Yu17}; a.o.), we still know relatively little about how (or whether) infixation {\it systematically} interacts with particular aspects of morphology and phonology. This paper presents a particularly revealing case study that contributes to a larger research program aiming to discover what systematic interactions there are (if any) between {\it infixation} and {\it allomorphy} across languages \citep{KalinIP}. These interactions have the potential to tell us a lot about the fine timing of the morphology-phonology interface, including when exponent choice happens, when exactly infixes ``become'' infixes, how often (re)prosodification happens during word building, and to what extent exponent choice and infixation may (or may not) be regulated by the phonology. %\citep{Kalin19,KalinIP}

The case study presented here is of Nancowry, a dialect of Nicobarese (Austroasiatic; Mon-Khmer) spoken by around 800 people on the island of Nancowry \citep[3]{Rad81}. The Nicobar Islands are a union territory of India, forming an arc (along with the Andaman Islands) between the Bay of Bengal and the Andaman Sea. The source for this case study is \citealt{Rad81}, a small grammar of the morphology and phonology of Nancowry, along with an extensive word list, based on fieldwork conducted in the early 1960s. (Hereafter all references with the format ``R:\#'' are page numbers from this work.) What makes this case study so informative is that there are several morphemes with exponents that are distributed based on the {\it prosodic size} of the stem they combine with, and some of these exponents are infixal, appearing in positions where they obscure earlier exponent choice and prosodification. 

\citet{Paster05,Paster06} briefly features Nancowry as an example of non-optimizing phonologically/prosodically-conditioned allomorphy. I confirm this finding and go significantly beyond it, showing that Nancowry provides evidence for the bottom-up cyclicity of exponent choice, infixation, and prosodification, applying in that  order within each cycle. These findings support a separation of morphology from phonology (see, e.g., \citealt{HalleMarantz93,HalleMarantz94,Trommer01,Paster06,Yu07,Embick10,BS12,Pak16,Dawson17,KalinIP,Kalin20,Rolle20,Stanton20}).

The paper is laid out as follows. \sectref{sec:kalin:2} presents a brief sketch of Nancowry's phonological and morphological system. \sectref{sec:kalin:3} investigates more deeply the two morphemes of interest -- the causative morpheme (\sectref{sec:kalin:3.1}) and the instrumental nominalizer (\sectref{sec:kalin:3.2}) -- and how they interact with each other (\sectref{sec:kalin:3.4}). \sectref{sec:kalin:4} turns to the theoretical implications of this data, and \sectref{sec:kalin:5} concludes.

\section{A sketch of Nancowry phonology and morphology}\label{sec:kalin:2}

Syllable structure plays a crucial role throughout Nancowry's phonological and morphological system, and so is a good entry point into understanding some basic properties of the language. All syllables in Nancowry have one of two shapes, CV or CVC, and syllabification ignores morphological structure (R:13--14). Stress is mostly predictable and is constrained to appearing only on root syllables. Roots, in turn, may be monosyllabic (CV or CVC; R:14) or disyllabic (CV.CV or CV.CVC; R:49); when monosyllabic, the sole root syllable bears stress, and when disyllabic, the second root syllable bears stress (R:15). The addition of other morphemes to a root/word never affects its stress pattern. 

Words (excluding those built via compounding and with particles) range from one to four syllables long. Examples of words of different sizes and of different morphological complexity, with stress placement indicated, are given in \Next.\footnote{As is traditional, infixes are indicated in angled brackets. In the gloss, the linear order of an infixal morpheme (with respect to other prefixes and infixes) corresponds to its (relative) closeness to the root in terms of selection and compositionality. The glosses {\sc onom, inom, anom} stand for objective nominalizer, instrumental nominalizer, agentive nominalizer, respectively. A full list of abbreviations is provided at the end of the paper.}$^{,}$\footnote{Since stress placement is mostly predictable, I omit stress marking in examples going forward. The only unpredictable aspect of the stress system is that stress is realized mainly on one vowel in a diphthong, and which vowel this is cannot be predicted -- it is lexically determined (R:15).}

\ea \label{kalinone}
\ea ka (fish) \hfill `fish' (R:93)
\ex l\'on (tame) \hfill `tame' (R:150)
\ex f\'aŋ-a (cut-{\sc onom}) \hfill `that which is cut' (R:135)
\ex ha-t\'əh ({\sc caus}-float) \hfill `to float something' (R:107)
\ex t<an>i\'an ({\sc <inom>}file) \hfill `a file'  (R:105)
\ex mil\'əh-a (play.a.game-{\sc onom}) \hfill `objects used in play' (R:147)
%\d. k<am><um>-tɯ\'a ({\sc <anom><caus>rp}-shaken) `earthquake' \hfill (R:106)
\ex p<am><um>l\'oʔ ({\sc <anom><caus>}loose) \hfill `one who loosens' (R:150)
\ex ma-ha-l\'ep-a ({\sc anom-caus}-fit-{\sc onom}) \hfill `a thing that is made to fit'  (R:45)
\z
\z

\noindent Words built via compounding and/or with particles may be longer than four syllables, with no clear upper size limit, e.g., {\it ɲ\'i-ma-ha-l\'iap-ta-ri} `school' (house-{\sc anom-caus-}know-{\sc ptcl-ptcl}  $\approx$ `house of one who makes you know') (R:117).

Stress placement constrains the distribution of phonemes in Nancowry. In stressed syllables, there are 10 phonemic vowel qualities, /i, e, ɛ, \ae, ɯ, ə, a, u, o, ɔ/ (R:24), 9 of which come in a (contrastive) nasalized variant (R:17), and 3 phonemic diphthongs, /ia, ua, ɯa/ (R:25). In unstressed syllables, only 3 vowel phonemes appear, /i, a, u/ (R:20), and neither diphthongs (R:24) nor nasalized vowels (R:17) are permitted. There are 16 consonant phonemes in Nancowry, /p, t, c, k, ʔ, m, n, ɲ, ŋ, f, s, h, r, l, w, j/ (R:33).\footnote{Note that /j/ is orthographically \textit{y} in all examples.} Consonants, unlike vowels, are not distributed based on whether a syllable bears stress or not, though unstressed {\it root} syllables (always of CV shape) come in a very restricted set of forms, including only the consonants /p, t, c, k, s, h, l/ (R:50). The only other major phonotactic constraint is that /r/ and /f/ cannot be codas (R:33). Consonant sequences of a wide variety occur, but only across syllable boundaries (R:36).

Nancowry has a small number of affixes, including some prefixes, infixes, and suffixes.\footnote{I put aside what the grammar calls ``particles'' here (see, e.g., R:47,82), as there is very little information given on this aspect of the morphological system.} The first affixes of interest are two non-productive components of Nancowry morphology that are tied closely to the root. Recall from above that roots come in four shapes, CV, CVC, CV.CV, and CV.CVC, with the initial CV of disyllabic roots being quite restricted in its form (R:50). For some apparent disyllabic roots, there is evidence that the initial CV syllable is actually a separable morpheme (called a ``root prefix''), though highly idiosyncratic and unproductive (R:48). Consider the following examples involving one such root prefix, \textit{ka-}:

\ea\label{kalinrps}  
\ea s\~ok `index finger' $\rightarrow$ {ka}-s\~ok `to give, to help' \hfill (R:133)
\ex hay `empty, air' $\rightarrow$ {ka}-hay `to feel empty (in the heart)' \hfill (R:127)
\ex yuaʔ `to pull out, remove' $\rightarrow$ {ka}-yuaʔ `to give birth' \hfill (R:156)
\ex yeʔ `to be afraid' $\rightarrow$ {ka}-yeʔ `wild (animal)' \hfill (R:49,152)\label{kalinafraid}
\z
\z

\noindent While \textit{ka-} typically signals a verbal word/stem, it does not always, cf.\ (\ref{kalinafraid}), and it neither makes a consistent semantic contribution nor combines with roots only of a certain category. Root prefixes never combine with disyllabic roots, and some (monosyllabic) roots may appear with more than one root prefix (though not at the same time).\footnote{I diverge from \citealt{Rad81} in my use of terminology in this paper. While the grammar refers to \textit{all} apparent disyllabic roots as consisting of a root prefix and a (monosyllabic) root (see R:48--50), I will only segment such disyllabic forms into two morphemes (a root prefix and a root) when there is evidence for this segmentation from related word/stem forms. For disyllabic forms with no such segmentation in evidence, I will simply treat these as true disyllabic, monomorphemic roots.} I will not attempt to formally account for the generalization that root prefixes combine only with monosyllabic roots, but speculate that it is due to a constraint on the maximum prosodic size (a foot) for the realization of this particular small chunk of morphosyntactic structure.

The second non-productive component of Nancowry morphology also takes monosyllabic roots and adds a prefix to build a disyllabic word/stem, this time with a (partially and opaquely) reduplicative affix (R:51--54). This so-called ``reduplicative prefix'' can be understood as having the (underlying) shape \textit{ʔiC},\footnote{The reduplicative prefix may actually be underlyingly \textit{iC}, with the glottal stop inserted phonetically to repair a vowel-initial word. This possibility is discussed more in \S\ref{kalinoptinf}.} with the coda of the prefix (\textit{C}) being a copy of the coda of the root, if there is one. However, a number of phonological alternations obscure this underlying form, including: (i)  the vowel of the reduplicative prefix surfaces as \textit{u} when there is (underlyingly, at least) a reduplicated coda in the prefix {\it and} this coda is non-coronal or an /l/, (ii) the coda in the reduplicative prefix is deleted {\it except} when it is a nasal or a non-glottal stop;\footnote{Coda deletion does {not} bleed the \textit{i/u} alternation, and so there is still evidence for the underlying reduplication process even when there is no overt coda consonant in the reduplicative prefix.} and (iii) surviving coda palatals become alveolar. (See \citealt[132ff.]{Steriade88} and \citealt[347ff.]{Alderete99} for the implications of this data for theories of reduplication.\footnote{Note that a number of other publications that discuss this reduplication pattern seem to be built on a misinterpretation of several basic components of the data, e.g., \citealt[247ff]{Hendricks99}, \citealt{Meek00}, and \citealt[223--224]{IZ05}. In particular, these works claim that ``the reduplicant does not have morphological meaning, but simply augments the verb'' \citep[58]{Hendricks99} in order to ``bring a stem up to the minimal size required for it to participate in another morphological construction'' \citep[200--201]{IZ05}. As the examples in (\ref{kalinreduppref}) show, the reduplicative prefix can make a morphological contribution (albeit an idiosyncratic, non-productive one). Further, there are no morphological constructions that depend on the presence of the reduplicative prefix; even the allomorphs that will be discussed in \sectref{sec:kalin:3} that combine {\it only} with disyllabic stems are not in general able to combine with stems containing the reduplicative prefix (R:55,61).}) Consider the examples in \Next, which illustrate the above processes.

\ea\label{kalinreduppref} 
\ea[]{ŋak `shine, bright' $\rightarrow$ {ʔuk}-ŋak `to flash' \hfill (R:116)}
\ex[]{tot `expensive' $\rightarrow$ {ʔit}-tot `to borrow' \hfill (R:110)}
\ex[]{hi `clean' $\rightarrow$ {ʔi}-hi  `to clear field for plantation' \hfill (R:124)}
\ex[]{miʔ `moist' $\rightarrow$ {ʔu}-miʔ `wet' \hfill (R:121)}
\ex[]{ruay `moving forward and backward' $\rightarrow$ {ʔi}-ruay `to beckon' \hfill (R:143)}
\ex[*]{luɲ (gap) $\rightarrow$ {ʔin}-luɲ `axe' \hfill (R:49)\label{kalinaxegap}}
\z
\z

\noindent Like with the root prefixes, the reduplicative prefix does not contribute a predictable meaning (and sometimes it contributes no meaning), though the derived form is often verbal. Some roots can appear with both a root prefix and a reduplicative prefix (though not at the same time), while other monosyllabic roots never take a reduplicative prefix, and yet others never appear \textit{without} the reduplicative prefix, like that in (\ref{kalinaxegap}). Finally, like with root prefixes, disyllabic roots cannot take a reduplicative prefix (R:49).\footnote{The grammar also states that both root prefixes and the reduplicative prefix can ``sometimes'' be dropped without informational loss in the presence of other affixes (R:49), but it was hard to confirm this with the available data.} The same speculation applies here as above, that there is a maximum output size on the realization of this deeply embedded root-related piece of morphosyntactic structure. Indeed, one might be tempted to simply treat the reduplicative prefix {\it as} a root prefix, but, whereas root prefixes freely co-occur with other  prefixes/infixes, the reduplicative prefix only does so in a restricted way (see, e.g., fn. \ref{kalinredfn}).

Beyond the two non-productive prefixes discussed above, there are several productive affixes in Nancowry. Nancowry has two suffixes, a possessive marker \textit{-a} (R:65) and an objective nominalizer \textit{-u} (R:66). There are also three productive prefixal/infixal morphemes, an agentive nominalizer (\textit{-am-}/\textit{m(a)-}; R:56--58), an instrumental nominalizer (\textit{-an-}/\textit{-in-}; R:60--64), and a causativize morpheme (\textit{ha-}/ \textit{-um-}; R:54--56). The latter two morphemes will be those of interest for the remainder of the paper.\footnote{I mostly put aside the agentive nominalizer because it is less clear how to analyze its forms and distribution, though I will occasionally bring this morpheme into the discussion.}

\section{Causatives and instrumental nominalizations}\label{sec:kalin:3}

This section investigates in detail the allomorphs of the causative morpheme and the instrumental nominalizer, as well as the interactions of these morphemes with each other. A thorough empirical characterization of the data sets the stage for understanding the theoretical implications of Nancowry morphophonology, which is taken up in \sectref{sec:kalin:4}.

\subsection{Causatives}\label{sec:kalin:3.1}

The causative morpheme has two suppletive allomorphs, whose properties are laid out in \Next (R:54--56) along with several examples. First, there is the prefix  \textit{ha-}, which combines only with monosyllabic stems, (\ref{kalincausallosa}). Next, there is the left-edge infix \textit{-um-}, for disyllabic stems, appearing after the first consonant of the stem, (\ref{kalincausallosb}). This morpheme derives verbs, most typically from adjectives, though occasionally from verbs and (rarely) from nouns.

\ea Allomorphs of the causative morpheme (first pass) \label{kalincausallos}
\ea \textit{ha-}\label{kalincausallosa}
\ea  Properties\\
$\bullet$ prefixal\\
$\bullet$ combines only with monosyllabic stems
\ex Examples\\
$\bullet$ pin `thick' $\rightarrow$ ha-pin `to thicken something' \hfill (R:111)\\
$\bullet$ ta `level' $\rightarrow$ ha-ta `to level something' \hfill (R:107)\\
$\bullet$ teh `to float' $\rightarrow$ ha-teh `to float something' \hfill (R:107)
\z
\ex \textit{-um-} \label{kalincausallosb}
\ea Properties\\
$\bullet$ infixal (appears after initial consonant; first vowel disappears)\\
$\bullet$ combines only with disyllabic stems
\ex Examples\\
$\bullet$ paloʔ `loose' $\rightarrow$ p<um>loʔ `to loosen' \hfill (R:150)\\
$\bullet$ tiyəh `new' $\rightarrow$ t<um>yəh `to make something new' \hfill (R:153)\\
$\bullet$ saput `to turn over' $\rightarrow$ s<um>put `to turn sthg over' \hfill (R:114)
\z
\z
\z

As can be seen in the examples in (\ref{kalincausallosb}), the infix \textit{-um-} overwrites the first vowel of the stem; thus, even though \textit{-um-} combines with disyllabic stems, the output is still disyllabic. This can be understood naturally if the ``phonological pivot'' \citep{Yu07} of the infix is the first vowel, with the infix placed {\it after} this pivot, such that infixation of \textit{-um-} creates vowel hiatus -- since complex vowels are not allowed in unstressed syllables (see \sectref{sec:kalin:2}), the stem vowel deletes. In other words, a form like \textit{p<um>loʔ} has an intermediate stage *\textit{pa<um>loʔ}.

In (\ref{kalincausallosb}), all provided examples involve unsegmentable disyllabic roots, but \textit{-um-} can also combine with segmentable disyllabic stems.\footnote{Note though that the causative only very rarely co-occurs with a reduplicative prefix. The grammar actually claims this is unattested entirely (R:55), but there are several exceptions in the word list. Exploring the reason for this rarity is outside the scope of this paper, though I have occasion to discuss one particular example of a causativized stem containing a reduplicative prefix in fn. \ref{kalinCCCfn}.\label{kalinredfn}} For example, the causative can combine with stems consisting of a root prefix and monosyllabic root (see \sectref{sec:kalin:2}, (\ref{kalinrps})), as shown in \Next.

\ea \ea fec (tiny) \hfill `tiny' (R:134)
\ex ka-fec ({\sc rp}-tiny) \hfill `to become tiny'\label{kalintiny}
\ex\label{kalincausrp} k<um>-fec ({\sc <caus>rp}-tiny) \hfill `to make something tiny'
\z
\z

\noindent In addition, causativization can recurse, resulting in a double causative, \Next.\footnote{Sometimes, both causative allomorphs appear, but there doesn't appear to be a doubly causative meaning (R:55--56). Perhaps in these cases \textit{ha-} is acting as a dummy root prefix of sorts.} (\ref{kalindoubcaus}) 
shows the infix \textit{-um-} combining with the already-causativized stem in (\ref{kalinsingcaus}).%Have the ``make someone fast'' example if needed

\ea \ea ʔ\~εh (near) \hfill `near'  (R:85)
\ex ha-ʔ\~εh ({\sc caus-}near) \hfill `to approach'\label{kalinsingcaus}
\ex\label{kalindoubcaus} h<um>-ʔ\~εh ({\sc <caus>caus-}near)  \hfill `to cause someone to approach'
\z
\z

\noindent Note that the double causative in (\ref{kalindoubcaus}) is built from a monosyllabic root, \textit{ʔ\~εh}. 

While it doesn't appear that disyllabic roots/stems (prior to any causative morpheme) can undergo double causativization (R:56), it is possible that such a double causative is simply phonologically invisible. Consider, for example, the disyllabic stem from (\ref{kalintiny}), \textit{kafec}. One application of the causative yields \textit{kumfec}, (\ref{kalincausrp}). If this were causativized a second time, the predicted outer causative allomorph would again be \textit{-um-} (because the stem is (still) disyllabic). Infixation of the second \textit{-um-} would yield (initially) *\textit{ku<um>mfec}, which would presumably be resolved back to \textit{kumfec} after the loss of the first stem vowel and the simplification of the illicit CCC sequence. In other words, there would be no surface evidence of the fact that there are (underlyingly) two instances of \textit{-um-}. This type of phonological explanation for a supposed morphological gap plays a role in understanding the interaction between the causative and instrumental nominalizer as well, as will be discussed in \sectref{sec:kalin:3.4}.

\subsection{Instrumental nominalizations}\label{sec:kalin:3.2}

The instrumental nominalizer also has two (main) suppletive allomorphs, laid out in \Next (R:60--64). As described in (\ref{kalininstallosb}), \textit{-an-} appears with monosyllabic stems and is infixal, surfacing after the first consonant and before the first vowel. On the other hand is \textit{-in-}, (\ref{kalininstallosa}), which appears with disyllabic stems and is also infixal, surfacing after the first consonant but overwriting the first vowel. This morpheme generally combines with verbs and derives an instrument noun.\footnote{I will mention here, but not pursue further, an alternative analysis of this allomorphy built on two suggestions by Heather Newell (p.c.), that (i) the stressed/unstressed vowel distinction in Nancowry corrresponds to a length distinction (where stressed vowels are bimoraic, unstressed vowels monomoraic), and (ii) \textit{-an-} and \textit{-in-} actually have the same infixal placement, namely, that they both want to follow the first vocalic mora. Pushing this one step further, it's even possible to posit one underlying form, \textit{-Vn-}. 

Here's how this analysis would capture the data at hand. When \textit{-Vn-} combines with a monosyllabic stem (which, by hypothesis, has a bimoraic nucleus) and tries to take a position after the first vocalic mora, it is blocked from doing so because geminates cannot be thus interrupted; instead, the infix ``repairs'' to a position preceding the long vowel. When \textit{-Vn-} combines with a disyllabic stem, the first vowel of the stem is monomoraic, and so \textit{-Vn-} is able to take a position after this vowel; as in the paper's proposed analysis, this position for the infix results in illegal vowel hiatus and deletion of the first stem vowel. Finally, the features of the underspecified vowel are determined by whether or not the \textit{n} in the infix is a coda (in which case the vowel is realized as assimilated front \textit{i}) or not (in which case the vowel is realized as unassimilated central \textit{a}). 

Since this alternative analysis of the allomorphy relies on prior stress assignment (feeding lengthening), it is still compatible with the general conclusions of the paper. And further, this explanation would not obviate the need for prosodically-conditioned suppletive allomorphy of the instrumental nominalizer entirely, cf.\ the discussion of \textit{-anin-} at the end of this section.\label{kalinfnheather}} 

\ea Allomorphs of the instrumental nominalizer (first pass) \label{kalininstallos}
\ea \textit{-an-} \label{kalininstallosb}
\ea Properties\\
$\bullet$ infixal (appears after initial consonant)\\
$\bullet$ combines only with monosyllabic stems\footnote{Cf.\ the discussion below about the so-called ``double instrumental''.}\\
\ex Examples\\
$\bullet$ rɯk `to arrive' $\rightarrow$ r<an>ɯk `vehicle' \hfill (R:138)\\
$\bullet$ tuak `to pull' $\rightarrow$ t<an>uak `thing to pull with' \hfill (R:109)\\
$\bullet$ kap `to bite' $\rightarrow$ k<an>ap `tooth' \hfill (R:61)
\z
\ex \textit{-in-} \label{kalininstallosa}
\ea  Properties\\
$\bullet$ infixal (appears after initial consonant; first vowel disappears)\\
$\bullet$ combines only with disyllabic stems\\
\ex Examples\\
$\bullet$ kasɯ `to trap' $\rightarrow$ k<in>sɯ `a trap' \hfill (R:130)\\
$\bullet$ calɯak `to swallow' $\rightarrow$ c<in>lɯak `throat' \hfill (R:146)\\
$\bullet$ tikoʔ `to prod' $\rightarrow$ t<in>koʔ `a prod' \hfill (R:97)
\z
\z
\z

The allomorph \textit{-in-} behaves much like the \textit{-um-} allomorph of the causative morpheme (\sectref{sec:kalin:3.1}): \textit{-in-} appears after the initial consonant of the stem, and the first vowel of the stem disappears, such that the output of infixation remains disyllabic. Like with \textit{-um-}, I propose that this is due to \textit{-in-} infixing after the first vowel, with the subsequent loss of that vowel to resolve hiatus. Also like  \textit{-um-},  \textit{-in-} can combine with morphologically complex stems consisting of a reduplicant and root (rare, with the same considerations as mentioned in fn. \ref{kalinredfn}) or root prefix and root (common), the latter shown in \Next.

\ea
\ea tal (cut.flesh) \hfill `to cut the flesh' (R:108)
\ex ki-tal ({\sc rp-}cut.flesh) \hfill `to saw (e.g., wood)' 
\ex k<in>-tal ({\sc <inom>rp-}cut.flesh) \hfill `a saw' \label{kalinrpinstc}
\z
\z
%\ex. \a. cuat (dig.with.fingers) \hfill `to dig with fingers'  (R:104)
%\b. ka-cuat ({\sc rp-}dig.with.fingers) \hfill `to dig a hole'
%\c. k<in>-cuat ({\sc <inom>rp-}dig.with.fingers) \hfill `scraper' \label{kalinrpinstc}

\noindent A discussion of the instrumental nominalizer combining with causativized stems (which are also disyllabic) is postponed to \sectref{sec:kalin:3.4}.

Unlike \textit{-in-} (and \textit{-um-}), the instrumental nominalizer allomorph \textit{-an-} appears after the first consonant of the stem and does \textit{not} supplant the first vowel of the stem; its phonological pivot for infixation, then, is simply the first consonant (or perhaps the first vowel, but preceding rather than following this vowel). While \textit{-an-} is generally seen combining with monosyllabic roots, as in the examples in (\ref{kalininstallosb}), it also seems to appear in the so-called ``double instrumental'' (R:63). \Next provides an example, with (\ref{kalindoubinstex}) segmented assuming there are two instrumental nominalizers in it (which I will question below).%Note that only stems that are (pre-nominalization) disyllabic have a double instrumental form. 

\ea\label{kalindoubinst} 
\ea kuac (trace) \hfill `a trace' (R:63,96)
\ex ta-kuac ({\sc rp-}trace) \hfill `to have a trace, to trace'
\ex t<in>-kuac ({\sc <inom>rp-}trace)  \hfill `an instrument to mark/trace' \label{kalinsinginstex}
\ex t<an><in>-kuac ({\sc <inom><inom>rp-}trace)  \hfill same as (\ref{kalinsinginstex}) \label{kalindoubinstex}
\z
\z

\noindent The apparent double instrumental, (\ref{kalindoubinstex}), consists of what looks like two instances of the instrumental nominalizer, but, there is no associated double instrumental meaning (in contrast to the double causatives of \sectref{sec:kalin:3.1}), and the double instrumental form is typically in free variation with a single instrumental form, (\ref{kalinsinginstex}) (again in contrast to double causatives).

What is particularly surprising here, if this is a true double instrumental, is that the instrumental nominalizer doesn't productively combine with nouns otherwise (which it must do in the hypothetical step from (\ref{kalinsinginstex}) to (\ref{kalindoubinstex})); further, even if this were possible, the allomorph of the instrumental nominalizer that would be expected given a disyllabic input like that in (\ref{kalinsinginstex}) is \textit{-in-}, not \textit{-an-}. A final puzzling feature of the supposed double instrumental is that stems that are (pre-nominalization) monosyllabic do not have a double instrumental form -- note that the double instrumental in \Last is formed on the basis of a disyllabic stem (a monosyllabic root plus a root prefix).\footnote{Unlike for the absence of double causatives of disyllabic stems (discussed at the end of \sectref{sec:kalin:3.1}), no phonological ``invisibility'' explanation of this gap is forthcoming. For example, if \textit{k<an>ap} from (\ref{kalininstallos}) took a second instrumental nominalizer, it would presumably have the hypothetical form *\textit{kinnap} (resolved from illicit *\textit{k<a<in>n>ap}), or perhaps the form *\textit{k<an><an>ap}, both of which are  phonologically well-formed in Nancowry.}

There are a number of possible analyses of the supposed double instrumental. The analysis that I take to be the best supported is that there is actually just a third allomorph of the instrumental nominalizer, \textit{-anin-}, which has the same distribution as the \textit{-in-} allomorph (first vowel as pivot; combines with disyllabic stems). Treating the ``double'' instrumental (synchronically, at least) as another suppletive allomorph of the instrumental morpheme would help explain all of its previously puzzling features -- the nominalizer doesn't need to be able to combine with a noun, no double instrumental meaning is expected, there is a natural way to understand the restriction to disyllabic stems, and it is easy to capture the free variation between \textit{-in-} and the ``double instrumental'' \textit{-anin-}.

The other possibilities for analyzing the double instrumental take there to be true double affixation, namely, infixation of \textit{-in-} followed by infixation of \textit{-an-}. Such analyses immediately face the challenge of why the second (outer)  nominalizer is \textit{-an-}, rather than \textit{-in-}, since the stem is disyllabic. There are a few of ways to try to explain this discrepancy. It may be that a morphological haplology constraint prohibits the adjacent identical allomorphs. Or, it may be that \textit{-an-} is the elsewhere allomorph, and \textit{-in-} is more restricted, e.g.: (i) \textit{-in-} will only combine with a CV.CVC stem; or (ii) \textit{-in-} will only combine with certain morphological kinds of disyllabic stems (crucially excluding ones containing a nominalizer already).\footnote{\citet[64]{Rad81} offers yet another potential explanation, namely, that after the first round of instrumental nominalization, the first syllable (\textit{tin} in (\ref{kalinsinginstex})) is reanalyzed as a monosyllabic root, thereby taking \textit{-an-} as the appropriate allomorph.}  However, solving this aspect of the morphological puzzle still would not explain why double affixation here does not have any semantic consequences, nor why only disyllabic stems can appear in the double instrumental. 

I therefore adopt the first entertained analysis, that there is no real double instrumental, and \textit{-anin-} is an additional suppletive allomorph of the instrumental nominalizer.

\subsection{Interim summary of allomorphy}\label{kalinallosec}\label{sec:kalin:3.3}

\Next and \NNext summarize the allomorphy exhibited by the causative and instrumental nominalizer, updated following the discussions in \S\S\ref{sec:kalin:3.1} and~\sectref{sec:kalin:3.2}, for easy reference. The next section turns to interactions between these morphemes/exponents.

\ea Allomorphs of the causative morpheme (updated; final version) \label{kalincausallos2}
\ea \textit{ha-}
\ea  prefixal
\ex combines only with monosyllabic stems
\z
\ex \textit{-um-} \label{kalincausallosb2}
\ea infixal (phonological pivot: follows first vowel)
\ex combines only with disyllabic stems\\~\\
\z
\z
\ex Allomorphs of the instrumental nominalizer (updated; final version)  \label{kalininstallos2}
\ea \textit{-an-} \label{kalininstallosb2}
\ea  infixal (phonological pivot: follows first consonant)
\ex combines only with monosyllabic stems
\z
\ex \textit{-in-} \label{kalininstallosa2}
\ea infixal (phonological pivot: follows first vowel)
\ex combines only with disyllabic stems
\z
\ex \textit{-anin-} \label{kalininstallosc2}
\ea infixal (phonological pivot: follows first vowel)
\ex combines only with disyllabic stems
\z
\z
\z

\subsection{Causative + Instrumental Nominalization}\label{sec:kalin:3.4}

The causative derives verbs, and the instrumental nominalizer takes verbs and derives nouns. Given these properties, we expect that the two morphemes should be able to combine, in particular, with the causative combining first with a root/stem, and then the instrumental nominalizer combining with the resulting verb. This is borne out, at least in part. 

Consider first what happens with monosyllabic roots. The causative allomorph expected with a monosyllabic root is the prefix \textit{ha-}, \sectref{sec:kalin:3.1}, resulting in a disyllabic word/stem. Given this derived disyllabic verb, when adding the instrumental nominalizer, the expected allomorph given stem size is the infix \textit{-in-}, \sectref{sec:kalin:3.2}. This is borne out, as seen in \Next.

\ea 
\ea ku\~at (curve) \hfill `curve'  (R:96)
\ex ha-ku\~at ({\sc caus-}curve) \hfill `to hang, to hook'
\ex\label{kalininstcaus} h<in>-ku\~at ({\sc <inom>caus-}curve) \hfill `a hook'
\z
\z

\noindent The infix \textit{-in-}, as expected, replaces the first vowel of the stem and appears after the first consonant in (\ref{kalininstcaus}). There are also attested examples (though fewer) showing that the variant \textit{-anin-} is allowed in instrumental nominalizations of the causative as well, in free variation with the allomorph \textit{-in-} as before (see (\ref{kalindoubinst})).

\ea \ea ru (make.shade) \hfill `shade' (R:67,141)
\ex ha-ru ({\sc caus-}make.shade) \hfill `to make shade'
\ex h<in>-ru ({\sc <inom>caus-}make.shade) \hfill `thing causing shade'
\ex h<anin>-ru ({\sc <inom>caus-}make.shade)\footnote{In the examples given by the grammar, these nominalizations have further undergone possessive marking (R:67); I have removed this additional marking for clarity of the point at hand.} \hfill `thing causing shade'
\z
\z

\noindent Thus far, then, all is as expected.

The wrinkle comes with stems that are (pre-causativization) disyllabic. The causative allomorph expected with a disyllabic stem is the infix \textit{-um-}, \sectref{sec:kalin:3.1}, resulting in a (still) disyllabic word/stem, e.g., \textit{saput $\rightarrow$ s<um>put} from (\ref{kalincausallosb}). Just as above, then, when adding the instrumental nominalizer, the expected allomorph is \textit{-in-}. Consider, however, what happens when you infix \textit{-in-} into a (complex) form like \textit{sumput} -- you derive the phonologically ill-formed *\textit{s<in>mput}. Logically speaking, this illicit CCC sequence might be resolved as either \textit{si{np}ut} or \textit{si{mp}ut}. Such CCC sequences elsewhere (though rare) are resolved by deletion of the medial consonant (i.e., the coda of the stem's first syllable);\footnote{This can be seen in the (rare) case of a causative infix appearing inside a reduplicative prefix that has a coda, where it is the infix's coda consonant which survives. For example, \textit{koɲ} `male' (R:97), whose form with the reduplicative prefix is \textit{ʔin-koɲ}, has the causative form \textit{ʔ<um>-koɲ} `to turn into a man' (R:97); the reduplicative prefix's coda, \textit{n}, is lost.\label{kalinCCCfn}} indeed, it's easy to confirm the absence of a \textit{si{mp}ut}-type resolution using the grammar's extensive word list. But, if CCC resolution gives us the former possibility, \textit{sinput}, then the derivationally prior causative \textit{-um-} infix has essentially disappeared entirely, and so there would be no (obvious) evidence that this is a nominalization of a causative in the first place. (Note that this is basically the same ``invisibility'' situation encountered in double causatives of disyllabic stems, as discussed at the end of \sectref{sec:kalin:3.1}.)

%in Radhakrishnan's (1981) fairly extensive word list, there are no \textit{CimC} sequences that could be understood as arising from the nominalization of a causative. 

Is there any evidence that there are, in fact, instrumental nominalizations of causatives built from disyllabic stems, despite their hypothesized surface invisibility? To try to answer this question, we can capitalize on the fact that the instrumental nominalizer only productively combines with verbs. If instrumental nominalizations of causatives built from disyllabic stems are in fact possible, then there should be cases of instrumental nominalizations that seem to take as their stem a {\it non}-verbal element, with meanings that semantically appear to incorporate a causative intermediate step (even though the causative affix is not visible inside it). There are indeed a number of such word forms, for example (\ref{kalininvisible1}) and (\ref{kalinice}):

\ea
\ea putoŋ `powder' \hfill (R:110)
\ex p<um>toŋ `to make powder'
\ex p<in>toŋ `white ant' (termite)\label{kalininvisible1}
\z
\z

\ea \ea sahuaŋ `cool' \hfill (R:40,63,128)
\ex s<um>huaŋ `to cool something' 
\ex s<in>huaŋ `something that cools, e.g., ice'\label{kalinice}
\z
\z

\noindent There are cases, too of the ``double causative'' \textit{-anin-} allomorph behaving similarly, e.g., \textit{lapəʔ} `beautiful' $\rightarrow$ \textit{l<anin>pəʔ} `aids to beautify' (R:112--113). It is of course possible that \textit{-in-/-anin-} in these cases is simply combining directly with a non-verbal element, as indeed \textit{-an-} occasionally does as well. But it is at least possible also that there is a surface-invisible causative in examples like (\ref{kalininvisible1}) and (\ref{kalinice}), lost when the instrumental nominalizer is added. 

\section{Discussion and implications}\label{sec:kalin:4}

\sectref{sec:kalin:3} covered in detail the properties and interactions of two morphemes whose allomorphs are crucially distributed based on the size of the stem that they combine with, the causative (\sectref{sec:kalin:3.1}; \textit{ha-} for monosyllabic stems, \textit{-um-} for disyllabic stems) and the instrumental nominalizer (\sectref{sec:kalin:3.2}; \textit{-an-} for monosyllabic stems and \textit{-in-} (or less commonly \textit{-anin-}) for disyllabic stems). This section turns to the implications of this data for the architecture of the morphology-phonology interface. 

Nancowry demonstrates the need for bottom-up cyclicity of exponent choice, infixation, and prosodification (syllable/foot construction). The idea that grammatical processes apply and may repeat in a bottom-up (smallest-to-largest constituent) fashion is a common assumption in many theories (see, e.g., \citealt{ChomskyHalle68,Kiparsky82,Kiparsky00,Carstairs87,Anderson92,Bobaljik00,Wolf08,Embick10,BS12}), but the cyclicity of infixation and its timing with respect to exponent choice and prosodification has not previously been  much discussed (though for some related discussions, see  \citealt[\S3.4.3]{Embick10}, \citealt{BF16}, and \citealt{Harizanov17}). The evidence for bottom-up cyclicity, elaborated and discussed below in \S\S\ref{sec:kalin:4.1}--\ref{sec:kalin:4.2}, comes from (i) considerations of what information must be present at different decision points in the derivation, and (ii) cases of opacity that emerge in the data. Nancowry also affords a window into the (non-)optimizing nature of allomorphy and infixation, as discussed in \sectref{sec:kalin:4.3}.

\subsection{Exponent choice and prosodification are cyclic}\label{sec:kalin:4.1}

Perhaps the most obvious implication of the Nancowry data is that phonological exponents of morphemes are chosen in a bottom-up, cyclic fashion. In the examples at hand, the most embedded element of the verbal complex is the verb root, and only once {\it its} phonological form is known can the right phonological form (exponent/allomorph) be chosen for the {\it next} layer of the morphological structure. This is true again at every structural level beyond the root -- for every morpheme whose phonological form is in question, the next-smaller constituent must first have a phonological form. 

Step-wise, bottom-up selection of exponents is perhaps most visible in the double causative, (\ref{kalindoubcaus}), and intrumental nominalizations of causatives, (\ref{kalininstcaus}), repeated below in \Next. (See the  discussions below (\ref{kalindoubcaus}) and  (\ref{kalininstcaus}) about what happens when the root/stem is disyllabic.)

\ea \label{kalin16}
\ea h<um>-ʔ\~εh ({\sc <caus>}{\sc caus-}near) \hfill  `to cause to approach'  (R:85)\label{kalin16a}
\ex h<in>-ku\~at  ({\sc <inom>}{\sc caus-}hang) \hfill `a hook' (R:96) \label{kalin16b}
\z
\z

\noindent To pick the right (inner) causative allomorph, the root's phonological form must be known (as well as the root prefix or reduplicative prefix, if there is one). To pick the right outer allomorph (a second causative, or the instrumental nominalizer), the (inner) causative's phonological form must be known, in combination with the root.

To be more precise here, it is not simply the {\it segmental} form of an inner constituent that needs to be visible for (outer) exponent choice, but rather its {\it prosodic size}: exponent choice in Nancowry relies on syllable count. Thus, there must also be cyclic (re-)prosodification at every node after exponent choice, establishing (minimally) syllable count, but potentially other prosodic structure as well. Returning to the stacked examples in \Last: there must be prosodification of the root exponent before the (inner) causative exponent is chosen (such that this inner morpheme can ``see'' whether its stem is monosyllabic or disyllabic), and then the causative must be prosodified with the root for the right outer affixal exponent to be chosen (such that the outer morpheme can, in turn, ``see'' whether its stem is monosyllabic or disyllabic). 

Evidence for bottom-up cyclicity of exponent choice also comes from opacity. In both (\ref{kalin16a}) and (\ref{kalin16b}), the choice of the \textit{ha-} allomorph is opaque: \textit{ha-} is selected on the basis of combining with a monosyllabic root, but after the infixation of the exponent of the next-outer morpheme (\textit{-um-} or \textit{-in-}), \textit{ha-} is no longer local to this conditioning environment. If infixation of (outer)  \textit{-um-}/\textit{-in-} were to have preceded exponent choice for the inner causative, then a different inner causative exponent would have been chosen (\textit{-um-}), on the basis of the stem being disyllabic (infix plus monosyllabic root). Similar evidence comes from the non-interference of an outer infix in the relationship between the root prefix and the root, \Next, examples repeated from (\ref{kalincausrp}) and (\ref{kalinrpinstc}):

\ea \label{kalin17}
\ea k<um>-fec ({\sc <caus>}{\sc rp-}tiny) \hfill  `to make something tiny'  (R:134)\label{kalin17a}
\ex k<in>-tal ({\sc <inom>rp-}cut.flesh) \hfill `a saw' (R:108)\label{kalin17b}
\z
\z

\noindent The root prefix must be in a very close selectional relationship with the root (\sectref{sec:kalin:2}), and even though on the surface, the infix appears between the root prefix and the root, this selectional relationship is not interrupted. 

Put in phonological rule terms, infixation counterfeeds/counterbleeds exponent choice (and other relationships) of/among more-embedded morphemes. In line with the findings discussed above, this means that the exponent of an inner affix is selected before that of an outer affix.

\subsection{When does infixation happen?}\label{sec:kalin:4.2}

The previous section discussed the evidence from Nancowry for bottom-up cycles of exponent choice and prosodification, but what about infixation? Infixation, too, is cyclic, and is ordered between exponent choice and prosodification within each cycle.

\subsubsection{Choice of an exponent choice precedes infixation (of that exponent)}\label{sec:kalin:4.2.1}

As can be seen in Nancowry, not all exponents of a morpheme have the same infixal status -- one may be a prefix while the other is an infix, as is the case for the allomorphs of the causative, and even two infixes might have different phonological pivots, as is the case for the allomorphs of the instrumental nominalizer. In other words, infixation is exponent-specific -- the right exponent must be chosen before it can be known whether the exponent should be infixed or not, and if so, what its infixal positioning is. This is true {\it even if} infixation in Nancowry is in part driven by optimization considerations -- as discussed in detail in \sectref{kalinopt}, there is still some degree of  arbitrariness to the phonological pivot that must be specified alongside each exponent.

The derivational priority of exponent choice over infixation of an exponent can be confirmed by opacity. \citet{KalinRolle21} note that the choice between \textit{-an-} and \textit{-in-} for the instrumental nominalizer is obscured in the derived surface forms, \Next, data repeated from (\ref{kalininstallos}).

\ea 
\ea k<an>ap `tooth' (<{\sc inom}>bite) \hfill (R:61)\label{kalin18a}
\ex  k<in>sɯ `a trap' (<{\sc inom}>trap) \hfill (R:130)\label{kalin18b}
\z
\z

\noindent In their infixed positions, both exponents are in disyllabic words and precede main stress; the basis on which the allomorphs are differentiated (stem size) is thus not immediately apparent in the surface form (what matters of course is the size of the stem \textit{prior} to infixation). Considering just the surface forms in \Last, the only difference between (\ref{kalin18a}) and (\ref{kalin18b}) that could be potentially leveraged for differentiating between the allomorphs is that one precedes a consonant, and one precedes a vowel. However, given that this very difference is a result of the two infixes having different phonological pivots, attempting to have exponent choice be governed by the infixed environment creates a chicken-and-egg problem. An independent problem with a surface-oriented analysis of this exponent choice (i.e., in an attempt to deny the derivational priority of exponent choice over infixation) is that, more  generally speaking, infixation {\it never} feeds exponent choice \citep{KalinIP}.  

Given that exponent choice for a morpheme is prior to infixation of that exponent, it is natural that the conditions that govern exponent choice should be independent from those that determine infix placement  \citep{KalinRolle21}. And indeed, this independence is demonstrated in Nancowry: as an example, the condition regulating the choice of \textit{-an-} as the exponent for the instrumental nominalizer is that the stem must be monosyllabic, while the condition on the placement of \textit{-an-} as an infix is that it should immediately follow the first consonant of the stem.

\subsubsection{Infixation precedes re-prosodification}\label{kalininford}\label{sec:kalin:4.2.2}

Once an infixal exponent is chosen, when does that infix get integrated phonologically and prosodically into its stem? There are two types of evidence in Nancowry that infixation happens within the same cycle as exponent choice (of the infix), and that the infix is in its surface infixed position prior to prosodification within that same cycle as well. 

The first relevant type of evidence, showing that infixation is ``immediate'', comes from agentive nominalizations of causatives built from disyllabic stems, like that in (\ref{kalin19c}).

\ea
\ea paloʔ (loose) \hfill `loose' (R:150)
\ex p<um>loʔ ({\sc <caus>}loose) \hfill `to loosen'
\ex p<am><um>loʔ ({\sc <anom><caus>}loose) \hfill `one who loosens something'\label{kalin19c}
\z
\z

\noindent It is not entirely clear what drives the choice of exponent for the agentive nominalizer as \textit{-am-} or \textit{m(a)-} (see R:56--58), though \textit{-am-} may combine with monosyllabic or disyllabic stems. What {\it is} clear is that \textit{-am-} has as its phonological pivot the first consonant of the stem, e.g.,  \textit{p<am>aloʔ} ({\sc <anom>}loose) `that which is loose'. Thus, in order for \textit{-am-} to appear in its attested position in (\ref{kalin19c}), \textit{-um-} must first have been placed into {\it its} infixal position. Since the two infixal exponents \textit{-um-} and \textit{-am-} have different phonological pivots, they cannot wait to be infixed at the same time without some additional stipulation about the pivot of the derived infixal complex. The form of (\ref{kalin19c}) follows straightforwardly so long as \textit{-um-} is infixed within the same cycle as exponent choice (of \textit{-um-}), and crucially {\it before} infixation of \textit{-am-} in the next cycle.

The second type of evidence for the ordering of infixation  is more tentative, and comes from double causatives of disyllabic stems and instrumental nominalizations of causatives of disyllabic stems. (See discussions at the end of \sectref{sec:kalin:3.1} and \sectref{sec:kalin:3.4} on the surface invisibility of the inner affix in these constructions.) Consider (\ref{kalin20a}), repeated from (\ref{kalininvisible1}), with its hypothesized structure given in (\ref{kalin20b}).

\ea
\ea p<in>toŋ \hfill `white ant' (termite) (R:110)\label{kalin20a}
\ex $[$ {\sc  inom $[$ caus} $[$ powder $]$$]$$]$\label{kalin20b}
\z
\z

\noindent Recall the explanation for the invisibility of the causative morpheme in (\ref{kalin20a}): the causative affix has the form \textit{-um-} (because the root \textit{putoŋ} is disyllabic); the outer morpheme, here the instrumental nominalizer, then appears in its \textit{-in-} form (because the derived stem \textit{p<um>toŋ} is disyllabic), and upon infixation, \textit{-in-} wipes out any phonological trace of \textit{-um-}. For this explanation to go through, it must be that at the point of exponent choice for the instrumental nominalizer, the inner causative exponent, \textit{-um-}, has already been infixed and prosodified as part of the stem. If it hadn't been, then the input to exponent choice for the instrumental nominalizer would be a trisyllabic form, consisting of (potentially unordered) components \textit{-um-} and  \textit{putoŋ}. A priori, we don't know what we'd expect the exponent of the instrumental nominalizer to be with a trisyllabic stem, but it's at least possible it would not be \textit{-in-}. Further, if \textit{-um-} had not already been infixed during the inner cycle, we'd face the problem of what to do with a sequence of infixes that should \textit{not} end up simply concatenated one after the other (like they happen to be in (\ref{kalin19c})).

Finally, one might wonder whether an infix could be placed simultaneous with prosodification (and potentially other phonological operations) within a cycle, rather than prior to prosodification. There are two arguments against simultaneity. First, recall that the phonological placement of \textit{-in-} and \textit{-um-} is opaque, as their phonological pivot -- the first vowel -- disappears; therefore, it must be that infix placement properly precedes at least vowel deletion.\footnote{The data are compatible with resolution of vowel hiatus being ``late''--either late within a cycle (after/during prosodification) or post-cyclic in the sense of applying only to the whole word.} Second, as will be elaborated in the next section, infix placement is not generally optimizing in Nancowry, and may even be anti-optimizing.

\subsubsection{Interim summary}\label{sec:kalin:4.2.3}

This section discussed the evidence from Nancowry that cycles are defined from the bottom-up, and that within a cycle, the ordering of operations is first (i) exponent choice, then (ii) infixation (if the exponent is infixal), and finally (iii) (re-)prosodification. For much additional data and incorporation of this ordering into a more complete model of the morphosyntax-phonology interface, see \citealt{KalinIP}.

\subsection{On optimization}\label{kalinopt}\label{sec:kalin:4.3}

There is a long tradition of using phonological optimization to explain both (i) patterns of phonologically-conditioned suppletive allomorphy (\citealt{MP93a,Mester94,Kager96,Mascaro96,Mascaro07,Wolf08,Kim10}, i.a.) and (ii) patterns of infixation (\citealt{MP93a,HI97,Horwood02,Wolf08}, i.a.). A natural question, then, is whether optimization is playing a role in the Nancowry data at hand. The answer is that optimization is at most playing a small role: exponent choice is for the most part not optimizing (and may even be anti-optimizing); and while there is a phonotactic motivation for moving certain exponents (once chosen) into the stem as infixes, a given exponent's precise infixed position inside the stem is largely arbitrary.

A preliminary reminder here is that there is no language-general disyllabic preference in Nancowry; see, e.g., the diverse set of words in (1). The {\it absence} of a general constraint on syllable count is amply evidenced throughout the language, including in the prefixal/infixal system itself: (i) the \textit{-anin-} allomorph of the instrumental nominalizer builds trisyllabic words/stems from disyllabic ones (see \sectref{sec:kalin:3.2}); and (ii) both allomorphs of the agentive nominalizer can build trisyllabic words/stems from disyllabic ones (see, e.g., \sectref{kalininford}). Recall from \sectref{sec:kalin:2} that there is also no {\it minimal} word size in Nancowry -- monosyllabic roots (even CV-shaped roots) are well-formed words. 

It is possible, however,  that there is a constraint on a certain very small piece of morphosyntactic structure (the root plus root prefix or reduplicative prefix) that it be maximally disyllabic (a foot), as suggested in \sectref{sec:kalin:2}. It is further possible to speculate that there is a derived environment effect at play (applying only to morphologically complex forms),  whereby the realization of this small  morphosyntactic structure is required to be {\it exactly} disyllabic. This small morphosyntactic domain, with a disyllabic constraint, may include the causative morpheme (in addition to root prefixes and the reduplicative prefix), but it crucially cannot include the instrumental or agentive nominalizers, nor any suffixes, which are very clearly not subject to any such restriction.


\subsubsection{Is exponent choice optimizing in Nancowry?}\label{sec:kalin:4.3.1}

When dealing with infixal exponents, it can be tricky to evaluate exponent choice independently from infixation in terms of optimization. To ask the question of whether {\it exponent choice} specifically is optimizing in Nancowry, I will consider each exponent as a complete package -- phonological form plus infixal placement. In \sectref{kalinoptinf}, I separately consider the extent to which {\it just} the placement (infixation) of the exponents is optimizing. To foreshadow the answer here, in agreement with the brief discussion of Nancowry by \citet[167--168]{Paster06}, exponent choice is not optimizing.

First consider the two causative allomorphs, \textit{ha-} and \textit{-um-} (\sectref{sec:kalin:3.1}). Is their distribution optimizing? Maybe, depending on which exponent you start from, and on what constraints you assume are active in their evaluation. I'll start by considering the causative exponent \textit{ha-}: \textit{ha-} is restricted to combining with monosyllabic stems (e.g., \textit{ha-pin} from (\ref{kalincausallos})), but it would be phonotactically absolutely fine for  \textit{ha-} to appear with a disyllabic stem (e.g., hypothetical *\textit{ha-saput} in place of attested \textit{s<um>put} from (\ref{kalincausallos})). In fact, \textit{ha-} is predicted to be {\it preferred} over \textit{-um-} given usual assumptions about optimization, because \textit{-um-} introduces a coda (a marked syllable structure) and causes vowel hiatus/deletion, in addition to \textit{-um-}  being an infix (a marked affix type that disrupts constituent integrity). From an optimization perspective, there thus does not seem to be any reason to choose \textit{-um-} over \textit{ha-} for disyllabic stems (though cf.\ the discussion below about how this would change if a disyllabic constraint is taken into account).

Now considering the opposite angle on the causative allomorphs: \textit{-um-} is restricted to combining with disyllabic stems (e.g., \textit{s<um>put}), and there is at least some reason to {\it not} choose this exponent with monosyllabic stems, i.e., \textit{-um-} is a bit worse with (some) monosyllabic stems than it is with disyllabic ones. Infixation of \textit{-um-} into a monosyllabic root would create both vowel hiatus (as it does even with disyllabic stems) and an illicit CC coda cluster, if the root has a coda (e.g., \textit{pin} from (\ref{kalincausallos}) would be hypothetical *\textit{p<um>n}, presumably resolvable as *\textit{pum}, cf.\ fn. \ref{kalinCCCfn}). However, this coda-cluster-avoidance explanation for choosing \textit{ha-} over \textit{-um-} does not extend to monosyllabic roots without a coda (e.g., \textit{ta} from (\ref{kalincausallos}) would have the hypothetical form *\textit{t<um>}, with no cluster problem). Further, avoidance of an illicit consonant sequence does not more generally motivate the choice of \textit{ha-} over \textit{-um-} -- if it could, we'd then predict \textit{ha-} to appear as the outer causative morpheme for double causatives of disyllabic stems, discussed at the end of \sectref{sec:kalin:3.1} (e.g., \textit{k<um>fec} from (\ref{kalincausrp}) would have the hypothetical double causative form *\textit{ha-k<um>fec}), rather than this double affixation being invisible.

The only way to salvage an optimizing characterization of the distribution of causative exponents \textit{ha-} and \textit{-um-} would be if there were a (derived environment) constraint preferring outputs that are exactly disyllabic (no smaller, no bigger) -- in such a case, indeed, \textit{ha-} would be best distributed with all monosyllabic stems, and \textit{-um-} with all disyllabic stems. However, as noted at the outset of \sectref{sec:kalin:4.3}, this constraint must be highly restricted to a small piece of morphosyntactic structure, and cannot apply, e.g., to the instrumental nominalizer allomorphs discussed below. So positing this constraint is only useful to a certain degree.

\begin{sloppypar}
Now consider the allomorphs of the instrumental nominalizer, \textit{-an-}, \textit{-in-}, and \textit{-anin-} (\sectref{sec:kalin:3.2}). The exponent that is restricted to monosyllabic stems, \textit{-an-} (e.g., \textit{k<an>ap} from (\ref{kalininstallos})), would be perfectly fine phonotactically on disyllabic stems as well (e.g., \textit{ta-kuak} from (\ref{kalindoubinst}) would be perfectly well formed as hypothetical *\textit{t<an>a-kuak}, rather than the attested \textit{t<in>-kuak} or \textit{t<anin>-kuac}). Indeed, choosing \textit{-an-} for disyllabic stems would avoid the vowel hiatus and coda introduced by \textit{-in-} and \textit{-anin-}, and so from an optimizing perspective, we expect \textit{-an-} to actually be preferred for all stems. This is similar to the case of causative \textit{ha-}, discussed above.
\end{sloppypar}

Starting instead from the perspective of the allomorphs \textit{-in-} and \textit{-anin-}, which are restricted to disyllabic stems, the picture is a little different. For monosyllabic stems with a coda, these forms would create an illicit CC coda cluster where the other allomorph, \textit{-an-}, would not (e.g., \textit{kap} would be hypothetical *\textit{k<in>p} or *\textit{k<anin>p}). This is like the case of causative \textit{-um-}, with one important exception:  because of the free variation between \textit{-in-} and \textit{-anin-}, no constraint preferring disyllabic outputs will help explain the distribution of the allomorphs of the instrumental nominalizer.\footnote{Note that there is also no reason that \textit{-an-} and \textit{-in-/-anin-} should not be completely swapped in their behavior, with \textit{-in-/-anin-} combining with monosyllabic stems and having as a pivot the first consonant, and \textit{-an-} appearing in disyllabic stems with the first vowel as its pivot \citep[167--168]{Paster06}. The only potentially optimizing aspect of their distribution is that the vowel \textit{i} appears before a coda coronal in \textit{-in-} (perhaps reflecting some kind of place assimilation, as also seen in the reduplicative prefix), while the \textit{n} of \textit{-an-} is an onset and so exerts no such pressure on its vowel. (See also fn.\ \ref{kalinfnheather}.) However, this is not a general constraint on the distribution of \textit{i} in Nancowry, and the only reason \textit{-in-} ends up as the rime of a syllable is because of its non-optimizing pivot, as will be discussed in \sectref{sec:kalin:4.3.2}.} Further, as discussed in the context of the causative allomorphs above, avoidance of an illicit consonant sequence does not seem to generally be able to motivate the choice of one allomorph over another. The evidence this time comes from the surface-invisible instrumental nominalizations of causatives of disyllabic stems discussed at the end of \sectref{sec:kalin:3.4}, where choosing \textit{-an-} as the instrumental nominalizer allomorph would make this construction surface visible and free of marked structures (e.g., producing a hypothetical form *\textit{p<an><um>toŋ} rather than the attested sub-optimal \textit{p<in>toŋ} from (\ref{kalininvisible1})/(\ref{kalin20a}), by hypothesis resolved from *\textit{p<u<in>m>toŋ}). 

In sum, exponent choice in Nancowry is not generally optimizing, and is sometimes even anti-optimizing. Exponent choice seems to be oblivious to phonotactic well-formedness considerations (at least for the instrumental nominalizer, though potentially also for the causative), even though these instances of exponent choice are prosodically conditioned. \citet{Paster05,Paster06} documents a number of other such cases of non-optimizing phonologically- and prosodically-conditioned allomorphy, and so this simply confirms her overall findings.

\subsubsection{Is infixation optimizing in Nancowry?}\label{kalinoptinf}\label{sec:kalin:4.3.2}

There are two ways to think about whether infixation in Nancowry is optimizing. First, given a particular exponent, is it optimizing for that exponent to be an infix, i.e., to not be a prefix? And second, given an infixal exponent, is its precise infixal position inside the stem phonologically optimizing?

The first question is easier to answer, though not wholly straightforward. The infixal exponents at hand, \textit{-um-}, \textit{-an-}, \textit{-in-}, and \textit{-anin-}, all have a vowel-initial shape, and as discussed in \sectref{sec:kalin:2}, all syllables must have an onset in Nancowry. It thus is indeed optimizing for these exponents to be infixes, since they thereby avoid creating an onsetless word.\footnote{\citet[\S2.5.1]{Yu07} calls this the ``ethological view of infixation,'' and notes its prevalence in the infixation literature; see, e.g., \citealt{Anderson72,Cohn92,Buckley97}.} This picture is complicated, however, by the fact that Nancowry arguably has another vowel-initial left-edge affix that is {\it not} infixal, the reduplicative prefix. In \sectref{sec:kalin:2}, I posited that the underlying shape of this prefix is \textit{ʔiC}. But, \citet[35]{Rad81} notes that it is not possible to tell whether glottal-initial words are underlyingly glottal initial, or whether such words are vowel initial and supplied with an initial glottal stop as a repair. \citet[348]{Alderete99} propose specifically that the reduplicative prefix is underlyingly vowel-initial, as evidenced by the fact that, when the agentive nominalizer \textit{m(a)-} combines with the reduplicative prefix, the initial glottal of the reduplicative prefix disappears, as in (\ref{kalin21b}).

\ea \ea ʔi-ti ({\sc red-}laugh) \hfill `to laugh' (R:58)
\ex m-i-ti ({\sc anom-red-}laugh) \hfill `one who laughs'\label{kalin21b}
\z
\z

\noindent The disappearance in (\ref{kalin21b}) of both the usually-present vowel \textit{a} in \textit{ma-} and of the reduplicative prefix's apparent glottal stop is easily explained if the reduplicative prefix is vowel-initial: prefixation of \textit{ma-} onto the vowel-initial stem \textit{i-ti} creates vowel hiatus, which is resolved by deletion of the first vowel. So, is it optimizing for \textit{-um-}, \textit{-an-}, \textit{-in-}, and \textit{-anin-} to be infixes rather than prefixes? Yes, but, there still must be something lexically-specified such that these vowel-initial affixes surface as infixes rather than prefixes with an initial glottal stop, in contrast to the vowel-initial reduplicative prefix.\footnote{Further, in the rare cases where an infix combines with a stem beginning with the reduplicative prefix, e.g., as seen in fn. \ref{kalinCCCfn}, there is not any obvious optimization-based motivation for the infix in such a case to move inside the stem rather than stay at the left edge -- no matter whether the infix is at the left edge or the reduplicative prefix is, the word will still be vowel initial.}

The second question, about whether infix {\it placement} is optimizing, is more complex. Consider first the instrumental nominalizer \textit{-an-} infix: \textit{-an-} appears after the initial consonant and before the first vowel, and this positioning will always produce a phonotactically well-formed stem/word; \textit{-an-} is {\it minimally} infixed. For \textit{-an-}, then, its infixal placement is straightforwardly a maximally optimizing solution for avoiding an onsetless word. 

For causative \textit{-um-} and instrumental nominalizer exponents \textit{-in-}/\textit{-anin-}, however, their positioning -- after the first vowel -- moves them gratuitously far inside the stem (in terms of achieving the goal of avoiding an onsetless word), introduces a word-internal coda, creates vowel hiatus, and results in an opaque surface form (since the infix's phonological pivot disappears); the placement of these infixes is thus {\it anti}-optimizing, causing more problems than it solves -- these infixes would all uniformly be more optimizing if they behaved like \textit{-an-} in their distribution. And again, recall from the beginning of \sectref{sec:kalin:4.3} that, at least for the instrumental nominalizer, it is not plausible to posit a constraint requiring disyllabic outputs, so this type of constraint cannot be a general motivating factor in infix placement in Nancowry. Finally, recall also that the infixal exponent of the agentive nominalizer, \textit{-am-}, {\it can} combine with disyllabic stems while (like \textit{-an-}) having the first consonant as its phonological pivot, so this configuration must not be ruled out by the language. 

In sum, if infixation were purely an optimization strategy in Nancowry, all VC(VC) left-edge affixes would be infixes (counter to fact) and all would have the initial consonant as their phonological pivot (counter to fact), modulo the caveat that causative \textit{-um-} might be subject to a disyllabic constraint, compelling its placement after the first vowel instead.

%Late addition of glottal stop supports post-first-V distribution of um/in b/c of what it looks like when um/in combines with reduplicative prefix

\subsubsection{Implications}\label{sec:kalin:4.3.3}

While it is tempting to analyze the distribution of exponents and their infixal nature in Nancowry as optimizing, a closer look shows that this is far from straightforward. Even if {\it parts} of the behavior of these exponents is optimizing, there is also a significant degree of arbitrariness involved, in particular, in the precise infixal position of the exponents and in terms of which vowel-initial affixes are infixes. This arbitrariness, as well as the opacity of the post-vowel placement of \textit{-um-} and \textit{-in-/-anin-}, would make it difficult to account for the alternations (both exponent choice and infixation) within the phonology proper. The type of approach that fits better with this data is one where both exponent choice and infixal position are  independent of the phonology, \`a la \citet{Paster06,Yu07,KalinIP,Kalin20,KalinRolle21}.%maybe add a fn about why Yu is not totally right.

\section{Conclusions}\label{sec:kalin:5}

In this paper, I have explored in depth the morphophonological behavior of two morphemes and their allomorphs in Nancowry, the causative morpheme and the instrumental nominalizer. This case study points to three core findings. First, exponent choice, infixation, and prosodification proceed cyclically from the most embedded morphosyntactic node up. Second, these three operations/processes apply serially within each cycle. And finally, exponent choice and infixation may be (together and separately) non-optimizing or even anti-optimizing, and so are not naturally regulated by the phonological component of the grammar, at least in this language.

The findings from Nancowry point to a separation of morphology from phonology (see, e.g.,  \citealt{Trommer01,Paster06,Yu07,Embick10,BS12,Pak16,Dawson17,Kalin20,Rolle20,Stanton20}), and are consistent with the results from investigating interactions between allomorphy and infixation in a sample of 40 languages \citep{KalinIP}, as well as from a broader view on conditions on exponent choice vs.\ exponent placement \citep{KalinRolle21}. While these findings may be accommodated in a number of morphological theories, they fit naturally within a Distributed Morphology late-insertion model (e.g., \citealt{HalleMarantz93,HalleMarantz94,Embick10}), with bottom-up exponent choice applying to the structure sent to spell-out, and each instance of exponent choice accompanied by some limited (morpho)phonological operations.

And so when {\it size matters} in infix allomorphy, we are afforded a unique window into the morphology-phonology interface.

%The findings from Nancowry are consistent with those from investigating interactions between allomorphy and infixation in a sample of more than 50 languages \citep{Kalin19,KalinIP}, as well as those from other in-depth case studies \citep{Kalin20}, as well as from a broader view on ``subcategorization''-based phenomena \citep{KalinRolle20}. 


\section*{Abbreviations}
\begin{multicols}{2}
\begin{tabbing}
{\sc onom}\hspace{.5ex}\= objective nominalizer\kill
{\sc anom} \> agentive nominalizer \\
{\sc caus} \> causative \\
{\sc inom} \> instrumental nominalizer\\
{\sc onom} \> objective nominalizer\\
{\sc ptcl} \> particle\\
{\sc rp}   \> root prefix\\
{\sc red}  \> reduplicative prefix\\
\end{tabbing}
\end{multicols}

\section*{Acknowledgements}

This paper has benefited from extensive discussions with Heather Newell and Nicholas Rolle. Thank you also to the helpful comments of several anonymous reviewers, of this work and related works. 

And of course, thank you to the star of this volume, Susi Wurmbrand. For the past many years, Susi has been an incredible source of inspiration, mentorship, and friendship (not to mention good wine). While the topic of this paper is somewhat outside of Susi's wheelhouse, it is very much inspired by the careful, exacting precision with which she treats empirical phenomena, the meaningful connections she draws across languages, and the way she expertly balances the big picture and the small picture. Susi, I hope this paper does you proud!

\printbibliography[heading=subbibliography,notkeyword=this]

\end{document}
